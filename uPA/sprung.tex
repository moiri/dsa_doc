\section{Sprung}
\label{uPA.sprung}
\textStyleAT{(Raufen)}
\begin{description}
    \item[Voraussetzung:]
        SF \textStyleSF{\nameref{sf.sprung}}
    \item[Probe:]
        PA+4
    \item[Wirkung:]
        Mit dieser Art der Parade, kann gegen einen bewaffneten Kämpfer pariert werden, ohne Waffenschaden hinnehmen zu müssen.
\end{description}
