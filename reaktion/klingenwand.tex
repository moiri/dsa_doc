\section{Klingenwand}
\label{reaktion.klingenwand}
\begin{description}
    \item[Voraussetzung:] SF \textStyleSF{\nameref{sf.klingenwand}}, Kämpfer führt keinen Schild, BE\textrm{ ${\leq}$ }4
    \item[Probe:]
        Zwei PA gegen zwei unterschiedliche Gegner, jeweils auf (PA/2)+2.\newline
        Bei Kenntnis der SF \textStyleSF{\nameref{sf.kampfgespuer}} können PA+4 Punkte frei auf zwei PA gegen zwei Gegner verteilt werden, wobei der Mindestwert einer PA 6 beträgt.\newline
        Bei Kenntnis der SF \textStyleSF{\nameref{sf.klingentaenzer}} können PA+6 Punkte frei auf drei PA gegen drei Gegner verteilt werden, wobei der Mindestwert einer PA 6 beträgt.\newline
        Ein \textStyleSF{\nameref{sf.klingentaenzer}} darf mit der PA-Aufteilung warten bis die Gegner ihre AT gwürfelt haben.
    \item[Wirkung:]
        Mit dieser Parade kann eine Kämpferin zwei Angriffe in direkter Folge parieren, also mit einem einzigen Parademanöver.
\end{description}
