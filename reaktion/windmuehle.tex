\subsection{Windmühle}
\label{reaktion.windmuehle}
\begin{description}
    \item[Voraussetzung]:
        SF \textStyleSF{\nameref{sf.windmuehle}}, Kämpfer trägt kein grosser oder sehr grosser Schild, der zu konternde Angriff ist ein \textStyleM{\nameref{aktion.wuchtschlag}}, \textStyleM{\nameref{aktion.hammerschlag}}, \textStyleM{\nameref{aktion.sturmangriff}} oder \textStyleM{\nameref{aktion.befreiungsschlag}}
    \item[Probe]:
        PA +8
    \item[Wirkung]:
        Dieses gewagte Manöver setzt die Wucht des gegnerischen Hiebes in eine eigene Angriffs-Aktion um.
        Gelingt die Windmühle, so gilt sie gleichzeitig als gelungener Wuchtschlag +8.
        Dieser kann mit einer Parade (keine Manöver möglich, verbraucht keine Aktion), erschwert um die Erschwernis des ursprünglichen Angriffmanövers (plus zusätzlich die Qualität der Windmühle) abgewehrt werden.
\end{description}
