\subsection{Beidhändiger Kampf}
\label{sf.beidhaendiger_kampf}
Mit der SF \textStyleSF{\nameref{sf.beidhaendiger_kampf} I} werden die Abzüge für den Kampf mit der falschen Hand auf -3 /-3 reduziert und erlaubt zusätzliche Manöver sowie die Nutzung des KK-Bonus auf die TP der Linken Hand.

Die SF \textStyleSF{\nameref{sf.beidhaendiger_kampf} II} erlaubt alle Abzüge für den Kampf mit der falsch Hand zu ignorieren und stellt eine zusätzlich Angriffs- oder Abwehr-Aktion zur Verfügung.

Die Regelung bezüglich der Anzahl Aktionen pro Kampfrunde ist im Kapitel \ref{chap.aktion.beidhaendiger_kampf} beschrieben.

\begin{description}
    \item[Voraussetzung]:
        GE 12, SF \textStyleSF{\nameref{sf.linkhand}} / GE 15, SF \textStyleSF{\nameref{sf.beidhaendiger_kampf} I}
    \item [Kosten]:
        100 AP (50 AP mit Vorteil \textStyleVT{Beidhändig}, 75 AP mit Vorteil \textStyleVT{Linkshändig}) / 400 AP (200 AP mit Vorteil \textStyleVT{Beidhändig}, 300 AP mit Vorteil \textStyleVT{Linkshändig}, allfällige Kosten von \textStyleSF{\nameref{sf.tod_von_links}} können angerechnet werden)
    \item [Referenz]:
        WDS 73
\end{description}
