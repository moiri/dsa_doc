\subsection{Rüstungsgewöhnung}
\label{sf.ruestungsgewoehnung}
Mit der SF \textStyleSF{\nameref{sf.ruestungsgewoehnung} I} sinkt die BE eines bestimmten Rüstungstyps um einen Punkt.

Mit der SF \textStyleSF{\nameref{sf.ruestungsgewoehnung} II} sinkt die BE aller Rüstungen um einen Punkt.

Mit der SF \textStyleSF{\nameref{sf.ruestungsgewoehnung} III} sinkt die BE aller Rüstungen um zwei Punkte.
Ausserdem wird nur die Hälfte der Behinderung vom INI-Wert abgezogen.

Der Einfluss der Initiative im Kampfgeschehen ist im Kapitel \ref{chap.INI} beschrieben.
\begin{description}
    \item[Voraussetzung]:
        KK 10 / KK 12, SF \textStyleSF{\nameref{sf.ruestungsgewoehnung} I} / KK 15, SF \textStyleSF{\nameref{sf.ruestungsgewoehnung} II}
    \item [Kosten]:
        150 AP (225 AP für Helden mit dem Vorteil \textStyleVT{Akademische Ausbildung (Magier)}) / 300 AP (450 AP für Helden mit dem Vorteil \textStyleVT{Akademische Ausbildung (Magier)}) / 450 AP (675 AP für Helden mit dem Vorteil \textStyleVT{Akademische Ausbildung (Magier)})
    \item [Referenz]:
        WDS 76
\end{description}
