\section{Eisenarm}
\label{sf.eisenarm}
\textStyleAT{(Raufen oder Ringen)}
Mit der SF \textStyleSF{\nameref{sf.eisenarm}} erleidet der Verteidiger bei einer gelungen Parade gegen einen Bewaffneten nur TP(A).
Er ist zudem in der Lage gegen Bewaffnete das Manöver \textStyleM{\nameref{bPA.binden}} und \textStyleM{\nameref{bPA.entwaffnen}} einzusetzen (wenn er die entsprechende SF besitzt).
Ein Kämpfer mit Eisenarm erleidet keine Abzüge auf seinen INI-Modifikator im Kampf gegen Bewaffnete.
\begin{description}
    \item[Voraussetzung:]
        KO 13
    \item [Kosten:]
        60 AP
    \item [Referenz:]
        WDS 91
\end{description}
