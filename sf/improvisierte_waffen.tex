\section{Improvisierte Waffen}
\label{sf.improvisierte_waffen}
Ein Kenner der SF \textStyleSF{\nameref{sf.improvisierte_waffen}} kann alle Mali im Nahkampf mit improvisierten Waffen ignorieren.
Im Fernkampf mit improvisierten Waffen gelten alle Mali auch bei Kenntnis der SF \textStyleSF{\nameref{sf.improvisierte_waffen}}.
Einzig der letzte Punkt (Patzer auch bei 19) kann ignoriert werden.
Die Waffen werden jedoch auch mit dieser SF nicht stabiler.

Nachfolgend die Mali für Kämpfer ohne Kenntnis dieser SF:

\begin{description}
    \item[Nahkampf] \hfill \\
        \begin{itemize}
            \item keine Manöver ausser Wuchtschlag (nur halbe Ansage als Schaden)
            \item Patzer auch bei 19, Prüfwurf um zusätzlich 5 erschwert
            \item Bruchfaktor Wurf bei jeder AT und PA
        \end{itemize}
    \item[Fernkampf] \hfill \\
        \begin{itemize}
            \item Die Zuschläge für die Zielgrösse steigen 3 Punkte anstatt um 2, ausgehend von 0 für sehr grosse Ziele
            \item Es können keine Manöver des Fernkampfs ausgeführt werden
            \item Patzer auch bei 19
        \end{itemize}
\end{description}

\begin{description}
    \item[Voraussetzung:]
        IN 12, TaW \textStyleTa{Raufen} 10, TaW \textStyleTa{Wurfwaffe} 10 wenn für Wurfwaffen verwendet
    \item [Kosten:]
        100 AP
    \item [Referenz:]
        WDS 74/98
\end{description}
