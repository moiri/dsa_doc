\section{Mercenario}
\label{uSpez.mercenario}
\begin{description}
    \item[Voraussetzung:]
        TaW Raufen 10, TaW Ringen 7
    \item[Kosten:]
        200 AP
    \item[Spezialisierung:]
        Raufen +1/+1
    \item[Verbilligte SFs:]
        \textStyleSF{\nameref{sf.auspendeln}}, \textStyleSF{\nameref{sf.beinarbeit}}, \textStyleSF{\nameref{sf.block}}, \textStyleSF{\nameref{sf.eisenarm}}, \textStyleSF{\nameref{sf.fussfeger}}, \textStyleSF{\nameref{sf.griff}}, \textStyleSF{\nameref{sf.knochenbrecher}}, \textStyleSF{\nameref{sf.schwinger}}, \textStyleSF{\nameref{sf.schwitzkasten}}, \textStyleSF{\nameref{sf.sprung}}, \textStyleSF{\nameref{sf.versteckte_klinge}}, \textStyleSF{\nameref{sf.wurf}}
    \item[WM gegen Bewaffnete:]
        0/-1
    \item[Speziell:]
        Die Manöver \textStyleSF{\nameref{bAT.unterlaufen} AT} und \textStyleSF{\nameref{bPA.unterlaufen} PA} sind um 4 erleichtert (Waffenlos oder mit Dolch).
        Ein Mercenario-Kämpfer ist in der Lage im bewaffneten Kampf, zusätzlich zu seiner Aktion und Reaktion einen weiteren Gegner mit dem Manöver Unterlaufen aus der AT am Angreifen zu hindern (die Waffen-AT und die Unterlaufen-AT müssen auf unterschiedliche Gegner geführt werden).

        Die Regelung bezüglich der Anzahl Aktionen pro Kampfrunde ist im Kapitel \ref{chap.aktion} beschrieben.
    \item [Referenz:]
        WDS 90
\end{description}
