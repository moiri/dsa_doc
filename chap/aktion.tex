% -*-mode: Latex-*-
% !TEX root = kampf.tex
% authors: simon maurer
%
% file: A.tex
% contents: main content of the document
% Sccs-Id: %W% %G%

%==============================================================================

\section[Zusätzliche Aktion/Reaktion]{\color{black} Zusätzliche
Aktion/Reaktion}
\label{bkm:RefHeading39621231957535}{\sffamily\color{black}
Generell gilt: Maximal sind pro Kampfrunde drei Aktionen bzw. Reaktionen
möglich.}

\subsection[Ausweichen]{\color{black} Ausweichen}
\label{bkm:RefHeading36231231957535}{\sffamily\color{black}
Ein gezieltes Ausweichen entspricht einer Parade (ebenfalls modifiziert
durch die Qualität des Angriffs). Mit der SF
\testStyleSF{Ausweichen I} und BE0, kann zusätzlich zu einer
Parade ausgewichen werden. Dasselbe gilt für die SF
\testStyleSF{Ausweichen II} und BE1, sowie
\testStyleSF{Ausweichen III} und BE2.}

\subsection[Schildkampf II]{\color{black} Schildkampf II}
{\sffamily\color{black}
Mit der Kenntnis der SF \testStyleSF{Schildkampf II} ist der
Kämpfer in der Lage, zusätzlich zu einer Aktion, ein Angriff mit seinem
Schild und ein weiterer Angriff mit seiner Hauptwaffe zu parieren. Dies
ist nur möglich mit BE\textrm{ ${\leq}$ }4.}

\subsection[Parierwaffen II]{\color{black} Parierwaffen II}
{\sffamily\color{black}
Mit der Kenntnis der SF \testStyleSF{Parierwaffen II} ist der
Kämpfer in der Lage zusätzlich zu einer Aktion, ein Angriff mit seiner
Parierwaffe und ein weiterer Angriff mit seiner Hauptwaffe zu
parieren.}

\subsection[Beidhändiger Kampf II]{\color{black} Beidhändiger Kampf II}
{\sffamily\color{black}
Mit der Kenntnis der SF \testStyleSF{Beidhändiger Kampf II} ist
der Kämpfer in der Lage zusätzlich zu einer Aktion und einer Reaktion
eine weitere Aktion oder Reaktion auszuführen.}

\subsection{Mercenario}
{\sffamily\color{black}
Mit der Kenntnis der Waffenlosen Spezialisierung
\testStyleSF{Mercenario} ist der Kämpfer in der Lage, im
bewaffneten Kampf zusätzlich zu seiner Waffen-Aktion und {}-Reaktion
das Manöver \testStyleSF{Unterlaufen aus der AT} auf einen
weiteren Gegner auszuführen (die Waffen-AT und die Unterlaufen-AT
müssen auf unterschiedliche Gegner geführt werden).}
\end{document}
