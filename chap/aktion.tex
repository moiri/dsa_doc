% -*-mode: Latex-*-
% !TEX root = kampf.tex
% authors: simon maurer
%
% file: aktion.tex
% contents: wann sind mehrere Aktionen möglich
% Sccs-Id: %W% %G%

%==============================================================================

\section{Zusätzliche Aktion/Reaktion}
\label{chap.aktion}
Generell gilt: Maximal sind pro Kampfrunde drei Aktionen bzw. Reaktionen möglich.

\subsection{Ausweichen}
\label{chap.aktion.ausweichen}
Ein gezieltes Ausweichen entspricht einer Parade (ebenfalls modifiziert durch die Qualität des Angriffs). Mit der SF \textStyleSF{\nameref{sf.ausweichen} I} und BE0, kann zusätzlich zu einer Parade ausgewichen werden.
Dasselbe gilt für die SF \textStyleSF{\nameref{sf.ausweichen} II} und BE1, sowie \textStyleSF{\nameref{sf.ausweichen} III} und BE2.

\subsection{Schildkampf II}
\label{chap.aktion.schildkampf}
Mit der Kenntnis der SF \textStyleSF{\nameref{sf.schildkampf} II} ist der Kämpfer in der Lage, zusätzlich zu einer Aktion, ein Angriff mit seinem Schild und ein weiterer Angriff mit seiner Hauptwaffe zu parieren.
Dies ist nur möglich mit BE\textrm{ ${\leq}$ }4.

\subsection{Parierwaffen II}
\label{chap.aktion.parierwaffen}
Mit der Kenntnis der SF \textStyleSF{\nameref{sf.parierwaffen} II} ist der Kämpfer in der Lage zusätzlich zu einer Aktion, ein Angriff mit seiner Parierwaffe und ein weiterer Angriff mit seiner Hauptwaffe zu parieren.

\subsection{Beidhändiger Kampf II}
\label{chap.aktion.beidhaendiger_kampf}
Mit der Kenntnis der SF \textStyleSF{\nameref{sf.beidhaendiger_kampf} II} ist der Kämpfer in der Lage zusätzlich zu einer Aktion und einer Reaktion eine weitere Aktion oder Reaktion auszuführen.

\subsection{Mercenario}
\label{chap.aktion.mercenario}
Mit der Kenntnis der Waffenlosen Spezialisierung \textStyleSF{Mercenario} ist der Kämpfer in der Lage, im bewaffneten Kampf zusätzlich zu seiner Waffen-Aktion und -Reaktion das Manöver \textStyleM{\nameref{aktion.unterlaufen} aus der AT} auf einen weiteren Gegner auszuführen (die Waffen-AT und die Unterlaufen-AT müssen auf unterschiedliche Gegner geführt werden).
