% -*-mode: Latex-*-
% !TEX root = kampf.tex
% authors: simon maurer
%
% file: wunden.tex
% contents: Beschreibung der Wunden und Wundschwellen
% Sccs-Id: %W% %G%

%==============================================================================
\section{Wunden}
%------------------------------------------------------------------
\subsection{Wundschwelle}
Jeder Kämpfer hat eine Wundschwelle.
Die Wundschwelle ist KO/2, allenfalls modifiziert durch den Vorteil Eisern (+2) oder den Nachteil Glasknochen (-2).
Zu Beginn eines Kampfes würfelt jeder Kämpfer eine Selbstbeherrschungsprobe, je fünf übrigbehaltene Punkte steigt die Wundschwelle für den Kampf um 1.

Wenn die erlittenen Schadenspunkte über der Wundschwelle liegen, erleidet der Kämpfer eine Wunde.
Dadurch sinken AT, PA, FK, AW, GE um je 2 Punkte und die GS um einen Punkt.

%------------------------------------------------------------------
\subsection{Heilung von Wunden}
Für unbehandelte Wunden gilt: Pro Ruhephase kann ein Verwundeter eine KO-Probe ablegen, die pro Wunde um drei Punkte erschwert ist.
Gelingt diese Probe, so heilt eine Wunde.

Für behandelte Wunden (Heilkunde Wunden +3 pro Wunde) gilt: Pro Ruhephase kann ein Verwundeter eine KO-Probe ablegen, die um die Hälfte der Tap* der Heilkunde Probe erleichtert ist.
Gelingt diese Probe, so heilt eine Wunde.
Weiter gilt: Je 7 TaP* aus der Heilkunde-Probe, bei 7 eingesetzten AsP eines Balsam und für je 7 LeP, die ein magischer Heiltrank zurück gibt, heilt eine Wunde sofort.

Jede Behandlung gilt nur für die folgende Ruhephase, kann aber vor jeder Ruhephase wiederholt werden.
