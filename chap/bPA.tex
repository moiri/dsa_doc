% -*-mode: Latex-*-
% !TEX root = kampf.tex
% authors: simon maurer
%
% file: bPA.tex
% contents: Attacken-Manöver des bewaffneten Kampfes
% Sccs-Id: %W% %G%

%==============================================================================
\section[Bewaffnete PA{}-Manöver]{\color{black} Bewaffnete PA-Manöver}
\label{bkm:RefHeading36211231957535}\subsection[Binden]{\color{black}
Binden}
{\sffamily\color{black}
\begin{description}\item[Voraussetzung]: SF \testStyleSF{Binden}}

\liststyleLiv
\begin{itemize}
\item {\sffamily\color{black}
Mit Hauptwaffe + \testStyleSF{Meisterparade}:\newline
Probe: +Ansage des Gegners, mindestens +4 + Ansage\newline
Wirkung: erleichtert die nächste Aktion um die Ansage und erhöht die
Qualität der nächsten Aktion um die Ansage.}
\item {\sffamily\color{black}
Mit \testStyleSF{Parierwaffe}:\newline
Probe: +Ansage des Gegners, mindestens +4\newline
Wirkung: erleichtert die nächste Aktion um 4 und erhöht die Qualität der
nächsten Aktion um 4}
\item {\sffamily\color{black}
Mit Hauptwaffe + \testStyleSF{Meisterparade} und
\testStyleSF{Parierwaffe}:\newline
Probe: +Ansage des Gegners, mindestens +4 +Ansage\newline
Wirkung: erleichtert nächste die Aktion um 4+Ansage und erhöht die
Qualität der nächsten Aktion um 4+Ansage}
\end{itemize}
\subsection[Defensiver Kampfstil]{\color{black} Defensiver Kampfstil}
{\sffamily\color{black}
\begin{description}\item[Voraussetzung]: SF \testStyleSF{Defensiver
Kampfstil}}

{\sffamily\color{black}
\item[Probe]: normale PA}

{\sffamily\color{black}
\item[Wirkung]: Zwei PA anstelle einer PA und einer AT.}

\subsection[Entwaffnen]{\color{black} Entwaffnen}
{\sffamily\color{black}
\begin{description}\item[Voraussetzung]: SF
\testStyleSF{Entwaffnen}}

{\sffamily\color{black}
\item[Probe]: PA+8}

{\sffamily\color{black}
\item[Wirkung]: Gelingt die PA des Verteidigers, muss der
Angreifer eine KK-Probe erschwert um 8 (um 10 bei Kenntnis der SF
\testStyleSF{Meisterliches Entwaffnen}) bestehen, sonst
verliert er die Waffe.}

{\sffamily\color{black}
\item[Besonderes]: Mit Bock, Hakendolch, Linkhand und ev.
Drachenklaue ist dieses Manöver um 2 Punkte erleichtert (benötigt
Kenntnis der SF \testStyleSF{Parierwaffen I}). Entwaffnen ist
üblicherweise nur gegen einhändig geführte Waffen möglich. Wer
allerdings die SF \testStyleSF{Meisterliches Entwaffnen}
beherrscht, kann auch Gegner mit Zweihandwaffen entwaffnen. Das
Vorgehen entspricht dem, was auch bei Einhandwaffen gilt.}

{\sffamily\color{black}
\item[Besonderes]: Eine zusätzliche Ansage kann als
KK-Erschwernis eingesetzt werden.}


\bigskip


\bigskip

\subsection[Formation]{\color{black} Formation}
{\sffamily\color{black}
\begin{description}\item[Voraussetzung]: SF \testStyleSF{Formation}}

{\sffamily\color{black}
\item[Probe]: normale PA}

{\sffamily\color{black}
\item[Wirkung]: Es ist möglich ohne Probenerschwernis für
einen Kameraden zu parieren, so sich dieser in unmittelbarer Nähe
befindet.}

\subsection[Gegenhalten]{\color{black} Gegenhalten}
{\sffamily\color{black}
\begin{description}\item[Voraussetzung]: SF
\testStyleSF{Gegenhalten}}

{\sffamily\color{black}
\item[Probe]: AT +4}

{\sffamily\color{black}
\item[Wirkung]: Derjenige mit der besseren Qualität macht
vollen Schaden, derjenigen mit der schlechteren nur den halben. Bei
gleicher Güte gewinnt der ursprüngliche Angreifer. Eine negative
Qualität wird wie üblich vom Schaden abgezogen (gegebenenfalls zuerst
halbieren).}

\subsection[Klingenwand]{\color{black} Klingenwand}
\label{bkm:RefHeading36811231957535}{\sffamily\color{black}
\begin{description}\item[Voraussetzung]: SF
\testStyleSF{Klingenwand}, Kämpfer führt keinen Schild,
BE\textrm{ ${\leq}$ }4}

{\sffamily\color{black}
\item[Probe]: Zwei PA gegen zwei unterschiedliche Gegner,
jeweils auf (PA/2)+2.\newline
Bei Kenntnis der SF \testStyleSF{Kampfgespür} können PA+4 Punkte
frei auf zwei PA gegen zwei Gegner verteilt werden, wobei der
Mindestwert einer PA 6 beträgt.\newline
Bei Kenntnis der SF \testStyleSF{Klingentänzer} können PA+6
Punkte frei auf drei PA gegen drei Gegner verteilt werden, wobei der
Mindestwert einer PA 6 beträgt.\newline
Ein \testStyleSF{Klingentänzer} darf mit der PA-Aufteilung
warten bis die Gegner ihre AT gwürfelt haben.}

{\sffamily\color{black}
\item[Wirkung]: Mit dieser Parade kann eine Kämpferin
zwei Angriffe in direkter Folge parieren, also mit einem einzigen
Parademanöver.}

\subsection[Meisterparade]{\color{black} Meisterparade}
{\sffamily\color{black}
\begin{description}\item[Voraussetzung]: SF
\testStyleSF{Meisterparade}, BE\textrm{ ${\leq}$ }4,
\testStyleSF{Me}\testStyleSF{i}\testStyleSF{sterparade}
mit Schild nur bei Kenntnis der SF \testStyleSF{Schildkampf
II}}

{\sffamily\color{black}
\item[Probe]: PA + Ansage}

{\sffamily\color{black}
\item[Wirkung]: Parade erschwert um Ansage ergibt eine
Erleichterung der nächsten Aktion in Höhe der Ansage.}


\bigskip


\bigskip


\bigskip


\bigskip

\subsection[Unterlaufen]{\color{black} Unterlaufen}
\label{bkm:RefHeading37211231957535}{\sffamily\color{black}
\begin{description}\item[Voraussetzung]: Distanzklasse der Waffe des
Verteidigers ist kürzer als jene des Angreifers}

{\sffamily\color{black}
\item[Probe]: PA+8}

{\sffamily\color{black}
\item[Wirkung]: Bei dieser waghalsigen Aktion nützt der
Verteidiger die Kürze seiner Waffe aus und unterläuft den Gegner. Die
nächste Reaktion des Gegners entfällt. Kennt der Gegner die SF
\testStyleSF{Halbschwert}, kann er trotz dem Unterlaufen
dennoch parieren (ohne Einsatz eines anderen Manövers).}

\subsection[Waffe zerbrechen]{\color{black} Waffe zerbrechen}
{\sffamily\color{black}
\begin{description}\item[Voraussetzung]: SF \testStyleSF{Waffe
zerbrechen}, nur gegen Klingenwaffen}

{\sffamily\color{black}
\item[Probe]: PA +8}

{\sffamily\color{black}
\item[Wirkung]: Wenn dem Verteidiger die PA gelingt, dann
muss er eine KK-Probe ablegen (Erschwernisse je nach Waffe, erleichter
um BF der gegnerischen Waffe). Bei Gelingen zerbricht die Waffe des
Gegners. Dieser hat die Möglichkeit seine Waffe fallen zu lassen um zu
verhindern, dass sie zerbricht.}

\subsection[Windmühle]{\color{black} Windmühle}
{\sffamily\color{black}
\begin{description}\item[Voraussetzung]: SF \testStyleSF{Windmühle},
Kämpfer trägt kein grosser oder sehr grosser Schild, der zu konternde
Angriff ist ein \testStyleSF{Wuchtschlag},
\testStyleSF{Hammerschlag}, \testStyleSF{Sturmangriff}
oder \testStyleSF{Befreiungsschlag}}

{\sffamily\color{black}
\item[Probe]: PA +8}

{\sffamily\color{black}
\item[Wirkung]: Dieses gewagte Manöver setzt die Wucht
des gegnerischen Hiebes in eine eigene Angriffs-Aktion um. Gelingt die
Windmühle, so gilt sie gleichzeitig als gelungener Wuchtschlag +8.
Dieser kann mit einer Parade (keine Manöver möglich, verbraucht keine
Aktion), erschwert um die Erschwernis des ursprünglichen
Angriffmanövers (plus zusätzlich die Qualität der Windmühle) abgewehrt
werden.}
