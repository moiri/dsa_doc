% -*-mode: Latex-*-
% !TEX root = kampf.tex
% authors: simon maurer
%
% file: uSF.tex
% contents: main content of the document
% Sccs-Id: %W% %G%

%==============================================================================

\chapter[Sonderfertigkeiten für den unbewaffneten Kampf]{\color{black}
Sonderfertigkeiten für den unbewaffneten Kampf}
{\sffamily\color{black}
Folgende Abbildung gibt eine Übersicht über alle Sonderfertigkeiten (mit
AP-Kosten und Voraussetzungen), die im unbewaffneten Kampf eingesetzt
werden können. Es handelt sich dabei nicht um die Manöver. Diese werden
in den Kapiteln 9 und 10 beschrieben. Hier gibt es einzelne
Abweichungen zum DSA Regelwerk 4.1.}


\bigskip



\begin{figure}
\centering
\includegraphics[width=16.976cm,height=12.437cm]{a110101kampfregelnskizze-img2.pdf}
\end{figure}
{\sffamily\color{black}
In den folgenden Unterkapitel werden Sonderfertigkeiten beschrieben, die
nicht Voraussetzung für ein bestimmtes Manöver sind, sondern einen
direkten Einfluss auf den Kämpfer und seine Fähigkeiten haben. Es
werden jedoch nur noch jene SFs beschrieben, die nicht bereits im
Kapitel 5 aufgelistet wurden.}

\section[Auspendeln (Raufen oder Ringen)]{Auspendeln
\textstyleEndnoteSymbol{(Raufen oder Ringen)}}
{\sffamily\color{black}
Mit der SF \testStyleSF{Auspendeln} ist der Verteidige in der
Lage, seinen Oberkörper scheinbar unabhängig von seinen Beinen zu
bewegen. Dies ist eine passive Sonderfertigkeit und erhöht den WM im
Kampf gegen Unbewaffnete um 0/+1.}

\section[Beinarbeit (Raufen oder Ringen)]{Beinarbeit
\textstyleEndnoteSymbol{(Raufen oder Ringen)}}
{\sffamily\color{black}
Mit der SF \testStyleSF{Beinarbeit} ist der Verteidiger in der
Lage auf sowohl sicheren wie auch Beweglichen Stand zu achten. Dies ist
eine passive Sonderfertigkeit und erhöht den WM im Kampf gegen
Unbewaffnete um 0/+1.}

\section[Eisenarm (Raufen oder Ringen)]{Eisenarm
\textstyleEndnoteSymbol{(Raufen oder Ringen)}}
{\sffamily\color{black}
Mit der SF \testStyleSF{Eisenarm} erleidet der Verteidiger bei
einer gelungen Parade gegen einen Bewaffneten nur TP(A). Er ist zudem
in der Lage gegen Bewaffnete das Manöver \testStyleSF{Binden}
und \testStyleSF{Entwaffnen} einzusetzen (wenn er die
entsprechende SF besitzt). Ein Kämpfer mit Eisenarm erleidet keine
Abzüge auf seinen INI-Modifikator im Kampf gegen Bewaffnete.}

\section[Versteckte Klinge (Raufen)]{Versteckte Klinge
\textstyleEndnoteSymbol{(Raufen)}}
{\sffamily\color{black}
Mit der SF \testStyleSF{Versteckte Klinge} ist der Kämpfer ist
in der Lage, eine Waffe mit der Distanzklasse Handgemenge mit seinen
Raufen Kampfwerten einzusetzen.}

\section{Wurf}
{\sffamily\color{black}
Mit der SF \testStyleSF{Wurf} ist der Kämpfer in der Lage, das
Manöver Niederschlagen/-werfen effektiver auszuführen.}
