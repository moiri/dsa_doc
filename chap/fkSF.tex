% -*-mode: Latex-*-
% !TEX root = kampf.tex
% authors: simon maurer
%
% file: fkSF.tex
% contents: main content of the document
% Sccs-Id: %W% %G%

%==============================================================================

\section[Sonderfertigkeiten für den Fernkampf]{\color{black}
Sonderfertigkeiten für den Fernkampf}
{\sffamily\color{black}
Folgende Abbildung gibt eine Übersicht über alle Sonderfertigkeiten (mit
AP-Kosten und Voraussetzungen), die im Fernkampf eingesetzt werden
können. Es handelt sich dabei nicht um die Manöver. Diese werden im
Kapitel 13 beschrieben.}


\bigskip



\begin{figure}
\centering
\includegraphics[width=16.93cm,height=6.454cm]{a110101kampfregelnskizze-img3.pdf}
\end{figure}
{\sffamily\color{black}
In den folgenden Unterkapitel werden Sonderfertigkeiten beschrieben, die
nicht Voraussetzung für ein bestimmtes Manöver sind, sondern einen
direkten Einfluss auf den Kämpfer und seine Fähigkeiten haben. Es
werden jedoch nur noch jene SFs beschrieben, die nicht bereits im
Kapitel 5 aufgelistet wurden.}

\subsection{Berittener Schütze}
{\sffamily\color{black}
Die SF \testStyleSF{Berittener Schütze} erlaubt es einem
Schützen die Waffe auf dem Reitpferd ohne zusätzlichen Zeitaufwand zu
spannen. Alle Aufschlage, mit denen ein Schuss oder Wurf vom sich
bewegenden Reittier aus belegt sind, können halbiert werden. Vor dem
Schuss oder Wurf muss keine Reiten-Probe abgelegt werden.}

\subsection{Improvisierte Waffen}
{\sffamily\color{black}
Die SF \testStyleSF{Improvisierte Waffen} kann eingeschränkt
auch für Wurfwaffen eingesetzt werden (muss nicht separat erlernt
werden). Ohne diese SF gelten folgende Regeln für das Werfen von
Improvisierten Wurfwaffen:}

\liststyleLvi
\begin{itemize}
\item {\sffamily\color{black}
Die Zuschläge für die Zielgrösse steigen 3 Punkte anstatt um 2,
ausgehend von 0 für sehr grosse Ziele}
\item {\sffamily\color{black}
Es können keine Manöver des Fernkampfs ausgeführt werden}
\item {\sffamily\color{black}
Patzer auch bei 19}
\end{itemize}
{\sffamily\color{black}
Der letzte Punkt kann bei Kenntnis der SF
\testStyleSF{Improvisierte Waffen} ignoriert werden}

\subsection[Meisterschütze]{Meisterschütze}
{\sffamily\color{black}
Mit der SF \testStyleSF{Meisterschütze} erleidet der Schütze /
Werfer keinen Aufschlag für einen \testStyleSF{Schnellschuss}
(ohne SF 2 Aktionen). Beim Manöver \testStyleSF{Ansage} kann
eine maximale Ansage in Höhe seines Fernkampfwerts (anstelle nur seines
Talentwerts) machen und es muss nur eine zusätzliche Aktion aufgewendet
werden.}

{\sffamily\color{black}
Ein \testStyleSF{Meisterschütze} ignoriert Zuschläge aus
Seitenwind und Steilschüssen.}

\subsection{Scharfschütze}
{\sffamily\color{black}
Mit der SF \testStyleSF{Scharfschütze} erleidet der Schütze /
Werfer nur einen Aufschlag von 1 Aktion (anstelle von 2 Aktionen) für
einen \testStyleSF{Schnellschuss}. Beim Manöver
\testStyleSF{Ansage}, kann die volle Ansage zu den TP addiert
werden und es werden zwei Aktionen weniger zum Zielen benötigt
(mindestens aber eine zusätzliche Aktion). Scharfschützen benötigen
beim Manöver \testStyleSF{Zielen} eine Aktion pro Punkt
reduzierte Erschwernis.}

\subsection[Schnellladen]{Schnellladen}
{\sffamily\color{black}
Die SF Schnellladen reduziert die Ladezeiten für Bögen um eine Aktion
(betragen aber mindestens 1 Aktion). Die SF kann auch für Armbrüste
erlernt werden: Es werden nur noch drei Viertel der angegebenen
Ladezeiten benötigt. Diese SF kann nur eingesetzt werden, wenn die BE
des Kämpfers 4 oder weniger beträgt.}

\section[Fernkampf Manöver]{Fernkampf Manöver}
\label{bkm:RefHeading43221231957535}\subsection[Ansage (kein Autotreffer
/ mehrere Aktionen)]{Ansage (kein Autotreffer / mehrere Aktionen)}
{\sffamily\color{black}
\begin{description}\item[Voraussetzung]: keine}

{\sffamily\color{black}
\item[Probe]: FK+Ansage}

{\sffamily\color{black}
\item[Wirkung]: Der Kämpfer führt einen besonders gut
gezielten Wurf / Schuss aus. Die TP werden um die Hälfte der Ansage
erhöht. Dabei müssend halb so viele zusätzliche Aktionen aufgewendet
werden wie der gewünschte Zuschlag beträgt.}

{\sffamily\color{black}
Bei Kenntnis der SF \testStyleSF{Scharfschütze }werden die TP um
die ganze Ansage erhöht und es müssen zwei Aktionen weniger zum Zielen
aufgewendet werden (mindestens jedoch eine zusätzliche Aktion).}

{\sffamily\color{black}
Bei Kenntnis der SF \testStyleSF{Meisterschütze} werden die TP
um die ganze Ansage erhöht und es muss nur eine Aktion zum Zielen
aufgewendet werden.}

\subsection[Eisenhagel (kein Autotreffer / Aktion)]{Eisenhagel (kein
Autotreffer / Aktion)}
{\sffamily\color{black}
\begin{description}\item[Voraussetzung]: SF
\testStyleSF{Eisenhagel}, nur mit Wurfsternen, -scheiben oder
-ringen}

{\sffamily\color{black}
\item[Probe]: FK+2*Anzahl der Geschosse (maximal 5) +5
pro zusätzlichem Ziel}

{\sffamily\color{black}
\item[Wirkung]: Es werden mehrere Geschosse gleichzeitig
geworfen. Ihre TP werden einzeln ermittelt.}

\subsection[Schnellschuss (kein Autotreffer / freie
Aktion)]{Schnellschuss (kein Autotreffer / freie Aktion)}
{\sffamily\color{black}
\begin{description}\item[Voraussetzung]: Wurfwaffe in der Hand, Schusswaffe
gespannt, Ziel nicht weiter als 10 Schritt entfernt}

{\sffamily\color{black}
\item[Probe]: FK+2}

{\sffamily\color{black}
\item[Wirkung]: Der Schuss / Wurf wird ungezielt auf das
Ziel abgegeben. Dadurch benötigt der Schütze / Werfer auch keine Aktion
zum Zielen. Bei Kenntnis der SF \testStyleSF{Scharfschütze} ist
die Probe nur um einen Punkt erschwert und bei Kenntnis der SF
\testStyleSF{Meisterschütze} entfällt der Aufschlag ganz.}

\subsection[Zielen (kein Autotreffer / mehrere Aktionen)]{Zielen (kein
Autotreffer / mehrere Aktionen)}
{\sffamily\color{black}
\begin{description}\item[Voraussetzung]: keine}

{\sffamily\color{black}
\item[Probe]: FK - Erleichterung}

{\sffamily\color{black}
\item[Wirkung]: Durch langes Zielen kann für zwei
zusätzlich aufgewendete Aktionen der Zuschlag auf die Fernkampf-Probe
um einen Punkt reduziert werden. Es können auf diese Art maximal 4
Punkte Erschwernis abgebaut werden. Bei Kenntnis der SF
\testStyleSF{Scharfschütze} oder
\testStyleSF{Meisterschütze} muss nur eine Aktion aufgewendet
werden um die Erschwernis um einen Punkt zu reduzieren.}
