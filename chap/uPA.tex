% -*-mode: Latex-*-
% !TEX root = kampf.tex
% authors: simon maurer
%
% file: uPA.tex
% contents: main content of the document
% Sccs-Id: %W% %G%

%==============================================================================

\chapter[Unbewaffnete PA{}-Manöver]{\color{black} Unbewaffnete
PA-Manöver}
\label{bkm:RefHeading41621231957535}\section[Block (Raufen oder
Ringen)]{Block \textstyleEndnoteSymbol{(Raufen oder Ringen)}}
{\sffamily\color{black}
\begin{description}\item[Voraussetzung]: SF \testStyleSF{Block},
BE\textrm{ ${\leq}$ }4}

{\sffamily\color{black}
\item[Probe]: PA + Ansage}

{\sffamily\color{black}
\item[Wirkung]: Dies ist das waffenlose Manöver der
\testStyleSF{Meisterparade}: Parade erschwert um Ansage ergibt
eine Erleichterung der nächsten Aktion in Höhe der Ansage.}

\section[Kreuzblock (Raufen)]{Kreuzblock
\textstyleEndnoteSymbol{(Raufen)}}
{\sffamily\color{black}
\begin{description}\item[Voraussetzung]: SF
\testStyleSF{Kreuzblock}}

{\sffamily\color{black}
\item[Probe]: PA+Ansage des Gegners, mindestens +4
+Ansage}

{\sffamily\color{black}
\item[Wirkung]: Dies ist das waffenlose Manöver des
\testStyleSF{Bindens} (Hauptwaffe und Parierwaffe): erleichtert
die nächste Aktion um 4+Ansage und erhöht die Qualität der nächsten
Aktion um 4+Ansage}

\section[Sprung (Raufen)]{Sprung \textstyleEndnoteSymbol{(Raufen)}}
{\sffamily\color{black}
\begin{description}\item[Voraussetzung]: SF \testStyleSF{Sprung}}

{\sffamily\color{black}
\item[Probe]: PA+4}

{\sffamily\color{black}
\item[Wirkung]: Mit dieser Art der Parade, kann gegen
einen Bewaffneten Kämpfer pariert werden, ohne Waffenschaden hinnehmen
zu müssen.}

\section[Unterlaufen (Raufen oder Ringen)]{Unterlaufen
\textstyleEndnoteSymbol{(Raufen oder Ringen)}}
{\sffamily\color{black}
\begin{description}\item[Voraussetzung]: Distanzklasse der Waffe des
Angreifers ist länger als Handgemenge}

{\sffamily\color{black}
\item[Probe]: PA+8}

{\sffamily\color{black}
\item[Wirkung]: Bei dieser waghalsigen Aktion nützt der
Verteidiger seine kurze Distanzklasse aus und unterläuft den Gegner.
Die nächste Reaktion des Gegners entfällt.}
