% -*-mode: Latex-*-
% !TEX root = kampf.tex
% authors: simon maurer
%
% file: uSpez.tex
% contents: Unbewaffnete Spezialisierungen
% Sccs-Id: %W% %G%

%==============================================================================


\chapter{Waffenloser Kampf}
\label{chap.uSpez}
Im waffenlosen Kampf gelten im Allgemeninen folgende Modifikatoren (einige Werte können durch Sonderfertigkeiten verbessert werden):

\begin{itemize}
    \item WM gegen Unbewaffnete: 0/-2 (Siehe \textStyleSF{\nameref{sf.beinarbeit}} und \textStyleSF{\nameref{sf.auspendeln}})
    \item WM gegen Bewaffnete: -1/-2
    \item Wundschwelle Gegner: +2 (Siehe \textStyleSF{\nameref{uSpez.hammerfaust}} und \textStyleSF{\nameref{uSpez.hruruzat}})
    \item INI gegen Bewaffnete: -2 (Siehe \textStyleSF{\nameref{sf.eisenarm}})
    \item PA gegen Bewaffnete, wenn PA gelungen: Halbe TP (Siehe \textStyleSF{\nameref{sf.eisenarm}})
    \item AT gegen Bewaffnete, wenn Gegner pariert: Halbe TP (nur Waffenschaden) (Siehe \textStyleSF{\nameref{sf.eisenarm}})
\end{itemize}

Der waffenlose Kampf erlaubt verschiedene Spezialisierungen, welche als Sonderfertigkeiten erworben werden können:

\section{Bornländisch}
\label{uSpez.bornlaendisch}
\begin{description}
    \item[Voraussetzung:]
        TaW Raufen 5, TaW Ringen 5
    \item[Kosten:]
        100 AP
    \item[Spezialisierung:]
        Ringen +1/+1
    \item[Verbilligte SFs:]
        \textStyleSF{\nameref{sf.auspendeln}}, \textStyleSF{\nameref{sf.block}}, \textStyleSF{\nameref{sf.fussfeger}}, \textStyleSF{\nameref{sf.griff}}, \textStyleSF{\nameref{sf.knochenbrecher}}, \textStyleSF{\nameref{sf.schwinger}}, \textStyleSF{\nameref{sf.schwitzkasten}}, \textStyleSF{\nameref{sf.wurf}}
    \item [Referenz:]
        WDS 89
\end{description}


\section{Gladiatorenstil}
\label{uSpez.gladiatorenstil}
\begin{description}
    \item[Voraussetzung:]
        TaW Raufen 7, TaW Ringen 7
    \item[Kosten:]
        150 AP
    \item[Spezialisierung:]
        Ringen oder Raufen +1/+1
    \item[Verbilligte SFs:]
        \textStyleSF{\nameref{sf.auspendeln}}, \textStyleSF{\nameref{sf.beinarbeit}}, \textStyleSF{\nameref{sf.block}}, \textStyleSF{\nameref{sf.doppelschlag}}, \textStyleSF{\nameref{sf.eisenarm}}, \textStyleSF{\nameref{sf.fussfeger}}, \textStyleSF{\nameref{sf.griff}}, \textStyleSF{\nameref{sf.knochenbrecher}}, \textStyleSF{\nameref{sf.kreuzblock}}, \textStyleSF{\nameref{sf.schwinger}}, \textStyleSF{\nameref{sf.sprung}}, \textStyleSF{\nameref{sf.sprungtritt}}, \textStyleSF{\nameref{sf.wurf}}, \textStyleSF{\nameref{sf.kreuzblock}}
    \item[Speziell:]
        Kann bis zu 3 Punkten weniger echten Schaden anrichten und stattdessen diese Punkte auf den Ausdauerschaden schlagen.
    \item [Referenz:]
        WDS 89
\end{description}

\section{Hammerfaust}
\label{uSpez.hammerfaust}
\begin{description}
    \item[Voraussetzung:]
        TaW Raufen 7
    \item[Kosten:]
        150 AP
    \item[Spezialisierung:]
        Raufen +1/+1
    \item[Verbilligte SFs:]
        \textStyleSF{\nameref{sf.auspendeln}}, \textStyleSF{\nameref{sf.block}}, \textStyleSF{\nameref{sf.doppelschlag}}, \textStyleSF{\nameref{sf.eisenarm}}, \textStyleSF{\nameref{sf.knochenbrecher}}, \textStyleSF{\nameref{sf.kreuzblock}}, \textStyleSF{\nameref{sf.schmetterschlag}}, \textStyleSF{\nameref{sf.schwinger}}
    \item[Wundschwelle Gegner:]
        0
    \item[Speziell:]
        Kann mit der blossen Faust massive Gegenstände beschädigen.
        TP(A) als Strukturschaden.
        Wenn der Kämpfer die SF Ausfall beherrscht kann, er dieses Manöver auch im waffenlosen Kampf anwenden.
    \item [Referenz:]
        WDS 89
\end{description}

\section{Hruruzat}
\label{uSpez.hruruzat}
\begin{description}
    \item[Voraussetzung:]
        TaW Raufen 10, TaW Ringen 7
    \item[Kosten:]
        200 AP
    \item[Spezialisierung:]
        Raufen +1/+1
    \item[Verbilligte SFs:]
        \textStyleSF{\nameref{sf.auspendeln}}, \textStyleSF{\nameref{sf.beinarbeit}}, \textStyleSF{\nameref{sf.block}}, \textStyleSF{\nameref{sf.doppelschlag}}, \textStyleSF{\nameref{sf.eisenarm}}, \textStyleSF{\nameref{sf.fussfeger}}, \textStyleSF{\nameref{sf.griff}}, \textStyleSF{\nameref{sf.knochenbrecher}}, \textStyleSF{\nameref{sf.kreuzblock}}, \textStyleSF{\nameref{sf.schwinger}}, \textStyleSF{\nameref{sf.sprung}}, \textStyleSF{\nameref{sf.sprungtritt}}, \textStyleSF{\nameref{sf.wurf}}
    \item[Wundschwelle Gegner:]
        0
    \item[Speziell:]
        Bei Attacken mit Ansage (mindestens 2) darf der Kämpfer mit 2W6 die TP(A) erwürfeln.
        Würfelt er ein Pash, so ist ihm ein Zat gelungen und er kann erneut 2W6 TP(A) austeilen (Die Ansage wird jedoch nur einmal addiert).
        Die Ansage kann im Verhältnis 2:1 benutzt werden um einen Würfel zu modifizieren.
    \item [Referenz:]
        WDS 90
\end{description}

\section{Mercenario}
\label{uSpez.mercenario}
\begin{description}
    \item[Voraussetzung:]
        TaW Raufen 10, TaW Ringen 7
    \item[Kosten:]
        200 AP
    \item[Spezialisierung:]
        Raufen +1/+1
    \item[Verbilligte SFs:]
        \textStyleSF{\nameref{sf.auspendeln}}, \textStyleSF{\nameref{sf.beinarbeit}}, \textStyleSF{\nameref{sf.block}}, \textStyleSF{\nameref{sf.eisenarm}}, \textStyleSF{\nameref{sf.fussfeger}}, \textStyleSF{\nameref{sf.griff}}, \textStyleSF{\nameref{sf.knochenbrecher}}, \textStyleSF{\nameref{sf.schwinger}}, \textStyleSF{\nameref{sf.schwitzkasten}}, \textStyleSF{\nameref{sf.sprung}}, \textStyleSF{\nameref{sf.versteckte_klinge}}, \textStyleSF{\nameref{sf.wurf}}
    \item[WM gegen Bewaffnete:]
        0/-1
    \item[Speziell:]
        Die Manöver \textStyleSF{\nameref{bAT.unterlaufen} AT} und \textStyleSF{\nameref{bPA.unterlaufen} PA} sind um 4 erleichtert (Waffenlos oder mit Dolch).
        Ein Mercenario-Kämpfer ist in der Lage im bewaffneten Kampf, zusätzlich zu seiner Aktion und Reaktion einen weiteren Gegner mit dem Manöver Unterlaufen aus der AT am Angreifen zu hindern (die Waffen-AT und die Unterlaufen-AT müssen auf unterschiedliche Gegner geführt werden).

        Die Regelung bezüglich der Anzahl Aktionen pro Kampfrunde ist im Kapitel \ref{chap.aktion} beschrieben.
    \item [Referenz:]
        WDS 90
\end{description}

\section{Unauer Schule}
\label{uSpez.unauer_schule}
\begin{description}
    \item[Voraussetzung:]
        TaW Ringen 10
    \item[Kosten:]
        150 AP (75 AP für Besitzer des Vorteils \textStyleSF{Schlangenmensch})
    \item[Spezialisierung:]
        Ringen +1/+1
    \item[Verbilligte SFs:]
        \textStyleSF{\nameref{sf.auspendeln}}, \textStyleSF{\nameref{sf.beinarbeit}}, \textStyleSF{\nameref{sf.block}}, \textStyleSF{\nameref{sf.eisenarm}}, \textStyleSF{\nameref{sf.griff}}, \textStyleSF{\nameref{sf.knochenbrecher}}, \textStyleSF{\nameref{sf.schwinger}}, \textStyleSF{\nameref{sf.schwitzkasten}}, \textStyleSF{\nameref{sf.wurf}}
    \item[Speziell:]
        Alle Entwinden-Manöver (GE-, Entfesseln-, Akrobatik- oder Ringen-Proben) sind für ihn um 2 erleichtert.
    \item [Referenz:]
        WDS 90
\end{description}

