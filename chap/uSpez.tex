% -*-mode: Latex-*-
% !TEX root = kampf.tex
% authors: simon maurer
%
% file: uSpez.tex
% contents: main content of the document
% Sccs-Id: %W% %G%

%==============================================================================


\section[Spezialisierungen Waffenloser Kampf]{\color{black}
Spezialisierungen Waffenloser Kampf}
{\sffamily\color{black}
Grundsätzlich gilt:}

\liststyleLv
\begin{itemize}
\item {\sffamily\color{black}
WM gegen Unbewaffnete: 0/{}-2}
\item {\sffamily\color{black}
WM gegen Bewaffnete: -1/-2}
\item {\sffamily\color{black}
Wundschwelle Gegener: +2}
\item {\sffamily\color{black}
INI gegen Bewaffnete: -2}
\item {\sffamily\color{black}
PA gegen Bewaffnete, wenn PA gelungen: Halbe TP}
\item {\sffamily\color{black}
AT gegen Bewaffnete, wenn Gegner pariert: Halbe TP (nur Waffenschaden)}
\end{itemize}
\subsection[Bornländisch]{\color{black} Bornländisch}
{\sffamily\color{black}
\item[Spezialisierung}: Ringen +1/+1}

{\sffamily\color{black}
\item[Verbilligte SFs}: Auspendeln, Block, Fussfeger,
Griff, Knochenbrecher, Schwinger, Schwitzkasten, Wurf}

\subsection[Gladiatorenstil]{\color{black} Gladiatorenstil}
{\sffamily\color{black}
\item[Spezialisierung}: Ringen oder Raufen +1/+1}

{\sffamily\color{black}
\item[Verbilligte SFs}: Auspendeln, Beinarbeit, Block,
Doppelschlag, Eisenarm, Fussfeger, Griff, Knochenbrecher, Kreuzblock,
Schwinger, Sprung, Sprungtritt, Wurf, Kreuzblock}

{\sffamily\color{black}
\item[Speziell}: Kann bis zu 3 Punkten weniger echten
Schaden anrichten und stattdessen diese Punkte auf den Ausdauerschaden
schlagen.}

\subsection[Hammerfaust]{\color{black} Hammerfaust}
{\sffamily\color{black}
\item[Spezialisierung}: Raufen +1/+1}

{\sffamily\color{black}
\item[Verbilligte SFs}: Auspendeln, Block, Doppelschlag,
Eisenarm, Knochenbrecher, Kreuzblock, Schmetterschlag, Schwinger}

{\sffamily\color{black}
\item[Wundschwelle Gegener}: 0}

{\sffamily\color{black}
\item[Speziell}: Kann mit der blossen Faust massive
Gegenstände beschädigen. TP(A) als Strukturschaden. Wenn der Kämpfer
die SF Ausfall beherrscht kann, er dieses Manöver auch im waffenlosen
Kampf anwenden.}


\bigskip


\bigskip


\bigskip


\bigskip


\bigskip


\bigskip


\bigskip

\subsection[Hruruzat]{\color{black} Hruruzat}
{\sffamily\color{black}
\item[Spezialisierung}: Raufen +1/+1}

{\sffamily\color{black}
\item[Verbilligte SFs}: Auspendeln, Beinarbeit, Block,
Doppelschlag, Eisenarm, Fussfeger, Griff, Knochenbrecher, Kreuzblock,
Schwinger, Sprung, Sprungtritt, Wurf}

{\sffamily\color{black}
\item[Wundschwelle Gegener}: 0}

{\sffamily\color{black}
\item[Speziell}: Bei Attacken mit Ansage (mindestens 2)
darf der Kämpfer mit 2W6 die TP(A) erwürfeln. Würfelt er ein Pash, so
ist ihm ein Zat gelungen und er kann erneut 2W6 TP(A) austeilen (Die
Ansage wird jedoch nur einmal addiert). Die Ansage kann im Verhältnis
2:1 benutzt werden um einen Würfel zu modifizieren.}

\subsection[Mercenario]{\color{black} Mercenario}
{\sffamily\color{black}
\item[Spezialisierung}: Raufen +1/+1}

{\sffamily\color{black}
\item[Verbilligte SFs}: Auspendeln, Beinarbeit, Block,
Eisenarm, Fussfeger, Griff, Knochenbrecher, Schwinger, Schwitzkasten,
Sprung, Verteckte Klinge, Wurf}

{\sffamily\color{black}
\item[WM gegen Bewaffnete}: 0/-1}

{\sffamily\color{black}
\item[Speziell}: Das Manöver
\testStyleSF{Unterlaufen} (AT und PA) ist für ihn um 4
erleichtert (Waffenlos oder mit Dolch). Ein Mercenario-Kämpfer ist in
der Lage im bewaffneten Kampf, zusätzlich zu seiner Aktion und Reaktion
einen weiteren Gegner mit dem Manöver Unterlaufen aus der AT am
Angreifen zu hindern (die Waffen-AT und die Unterlaufen-AT müssen auf
unterschiedliche Gegner geführt werden).}

{\sffamily\color{black}
Die Regelung bezüglich der Anzahl Aktionen pro Kampfrunde ist im Kapitel
14 beschrieben.}

\subsection[Unauer Schule]{\color{black} Unauer Schule}
{\sffamily\color{black}
\item[Spezialisierung}: Ringen +1/+1}

{\sffamily\color{black}
\item[Verbilligte SFs}: Auspendeln, Beinarbeit, Block,
Eisenarm, Griff, Knochenbrecher, Schwinger, Schwitzkasten, Wurf}

{\sffamily\color{black}
\item[Speziell}: Alle Entwinden-Manöver (GE-,
Entfesseln-, Akrobatik- oder Ringen-Proben) sind für ihn um 2
erleichtert.}
