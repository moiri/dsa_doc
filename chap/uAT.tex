% -*-mode: Latex-*-
% !TEX root = kampf.tex
% authors: simon maurer
%
% file: uAT.tex
% contents: main content of the document
% Sccs-Id: %W% %G%

%==============================================================================

\chapter[Unbewaffnete AT{}-Manöver]{\color{black} Unbewaffnete
AT-Manöver}
\label{bkm:RefHeading41601231957535}\section[Doppelschlag
(Autotreffer / Raufen{}-Aktion)]{Doppelschlag
\textstyleEndnoteSymbol{(Autotreffer /
}\textstyleEndnoteSymbol{Raufen-}\textstyleEndnoteSymbol{Aktion)}}
{\sffamily\color{black}
\begin{description}\item[Voraussetzung]: SF
\testStyleSF{Doppelschlag}}

{\sffamily\color{black}
\item[Probe]: AT+2 pro Hand, Zweithand zusätzlich +3 ohne
\testStyleSF{Beidhändiger Kampf II}}

{\sffamily\color{black}
\item[Wirkung]: Dies ist das waffenlose Manöver des
\testStyleSF{Doppelangriff}\testStyleSF{s}: Mit beiden
Händen wird gleichzeitig zugeschlagen. Es werden zwei Paraden benötigt
(PA und Gezieltes Ausweichen, PA und zusätzliche PA von Zweithand).}

\section[Fussfeger (kein Autotreffer / Raufen{}-Aktion)]{Fussfeger
\textstyleEndnoteSymbol{(kein Autotreffer / Raufen-Aktion)}}
{\sffamily\color{black}
\begin{description}\item[Voraussetzung]: SF \testStyleSF{Fussfeger}}

{\sffamily\color{black}
\item[Probe]: AT+8 (+F)}

{\sffamily\color{black}
\item[Wirkung]: Dies ist das waffenlose Manöver des
\testStyleSF{Umreissens}: Kann nicht Pariert werden, muss das
Opfer eine um die TP(A) des Angriffs erschwerte GE Probe bestehen um
auf den Beinen zu bleiben (Erleichterungen für
\testStyleSF{Standfest} (-2), \testStyleSF{Balance}
(-4) und \testStyleSF{Herausragende Balance} (-8)). Ein Treffer
erzeugt keinen Schaden.}

{\sffamily\color{black}
\item[PA für den Gegner}: Der Angriff kann nur mit einer
Waffen-PA+8, einer Raufen- oder Ringen-PA mit
\testStyleSF{Beinarbeit} oder mit einem
\testStyleSF{Ausweichen} pariert werden.}

{\sffamily\color{black}
\item[Besonderes]: Kann mit einem
\testStyleSF{Schwinger} kombiniert werden.}

\section[Griff (kein Autotreffer / Ringen{}-Aktion)]{Griff
\textstyleEndnoteSymbol{(}\textstyleEndnoteSymbol{kein Autotreffer /
}\textstyleEndnoteSymbol{Ringen}\textstyleEndnoteSymbol{{}-Aktion}\textstyleEndnoteSymbol{)}}
{\sffamily\color{black}
\begin{description}\item[Voraussetzung]: SF \testStyleSF{Griff}}

{\sffamily\color{black}
\item[Probe]: AT +Ansage (+F)}

{\sffamily\color{black}
\item[Wirkung]: Ein Griff soll den Gegner kampfunfähig
machen und zielt nicht darauf ab, Schaden zu verursachen. Kann der
Angriff nicht Pariert werden, ist jede folgende Aktion des Gegriffenen
um die doppelte Ansage erschwert. Der Griff hält an, bis der Greifende
freiwillig loslässt oder der Gegriffene sich befreien kann: Der
Gegriffene kann versuchen, den Griff mit einer (nicht erschwerten)
Raufen oder Ringen-AT abzuschütteln, die wiederum vom Greifenden mit
einer normalen Ringen-PA pariert werden kann.}

{\sffamily\color{black}
\item[Besonderes]: Kann mit einem
\testStyleSF{Schwinger} kombiniert werden.}


\bigskip


\bigskip


\bigskip


\bigskip


\bigskip

\section[Knochenbrecher (Autotreffer / Raufen{}- oder
Ringen{}-Aktion)]{Knochenbrecher \textstyleEndnoteSymbol{(Autotreffer /
}\textstyleEndnoteSymbol{Raufen-}\textstyleEndnoteSymbol{
}\textstyleEndnoteSymbol{oder
Rin}\textstyleEndnoteSymbol{g}\textstyleEndnoteSymbol{en-Aktion}\textstyleEndnoteSymbol{)}}
{\sffamily\color{black}
\begin{description}\item[Voraussetzung]: keine}

{\sffamily\color{black}
\item[Probe]: AT+Ansage (+F)}

{\sffamily\color{black}
\item[Wirkung]: Dies ist das waffenlose Manöver des
\testStyleSF{Wuchtschlags}: Der Kämpfer führt einen besonders
kraftvollen Schlag aus. Die TP(A) werden um die Hälfte der Ansage
erhöht. Bei Kenntnis der SF \testStyleSF{Knochenbrecher} werden
die TP gar um die ganze Ansage erhöht.}

{\sffamily\color{black}
\item[Besonderes]: Kann mit einem
\testStyleSF{Schwinger} kombiniert werden.}

\section[Niederschlagen/{}-werfen (Autotreffer / Raufen{}- oder
Ringen{}-Aktion)]{Niederschlagen/-werfen
\textstyleEndnoteSymbol{(}\textstyleEndnoteSymbol{Autotreffer /
}\textstyleEndnoteSymbol{Raufen- oder
Ringen}\textstyleEndnoteSymbol{{}-}\textstyleEndnoteSymbol{Aktion}\textstyleEndnoteSymbol{)}}
{\sffamily\color{black}
\begin{description}\item[Voraussetzung]: SF
\testStyleSF{Knochenbrecher}}

{\sffamily\color{black}
\item[Probe]: AT+4(+Ansage) (+WS)}

{\sffamily\color{black}
\item[Wirkung]: Dies ist das waffenlose Manöver des
Niederwerfens. Das Manöver verursacht Schaden wie üblich. Zusätzlich
muss der Gegner eine KK-Probe, erschwert um eine allfällige Ansage,
Erleichterungen für \testStyleSF{Standfest} (-1),
\testStyleSF{Balance} (-2), \testStyleSF{Herausragende
Balance} (-4) bestehen um auf den Beinen zu bleiben. Beherrscht der
Angreifer die SF \testStyleSF{Wurf }(Nur Ringen) und wurde der
Verteidiger zuvor mit dem Manöver \testStyleSF{Griff}
gegriffen, so steht dem Verteidiger keine Probe zum Stehenbleiben zu.}

{\sffamily\color{black}
\item[Besonderes]: Kann mit einem
\testStyleSF{Sprungtritt} und einem
\testStyleSF{Knochenbrecher} kombiniert werden.}

\section[Passierschlag (kein Autotreffer / Freie Raufen oder
Ringen{}-Aktion)]{Passierschlag \textstyleEndnoteSymbol{(kein
Autotreffer / Freie Raufen oder Ringen-Aktion)}}
{\sffamily\color{black}
\begin{description}\item[Voraussetzung]: Ein Kämpfer durchschreitet den
eigenen Kontrollbereich oder gibt sich durch eine Aktion eine Blösse}

{\sffamily\color{black}
\item[Probe]: AT+4}

{\sffamily\color{black}
\item[Wirkung]: Als Freie Aktion kann einem
unvorsichtigen Kämpfer, der den eigenen Kontrollbereich durchschritten
oder sich durch ein Manöver in eine unglückliche Lage begeben hat
(Sturmangriff) ein ungezielter Schlag versetzt werden (keine Manöver).
Besitzt das Opfer die SF \testStyleSF{Aufmerksamkeit} ist der
Schlag um weitere 4 Punkte erschwert, bei der SF
\testStyleSF{Kampfgespür} gar um 6 Punkte (4 aus
\testStyleSF{Aufmerksamkeit} und 2 aus
\testStyleSF{Kampfgespür}).}

\section[Schmetterschlag (Autotreffer /
Raufen{}-Aktion)]{Schmetterschlag \textstyleEndnoteSymbol{(Autotreffer
/ }\textstyleEndnoteSymbol{Raufen-}\textstyleEndnoteSymbol{Aktion)}}
{\sffamily\color{black}
\begin{description}\item[Voraussetzung]: SF
\testStyleSF{Schmetterschlag}}

{\sffamily\color{black}
\item[Probe]: AT+2}

{\sffamily\color{black}
Wirkung: Dies ist das waffenlose Manöver des
\testStyleSF{Betäubungsschlags}: Ein starker Schlag, der den
Gegner bewusstlos schlagen soll. Übersteigen die TP(A) die Wundschwelle
des Gegners, muss dieser eine KO-Probe ablegen. Bei Misslingen, fällt
das Opfer für 1W6 SR in Ohnmacht. Übersteigen die TP(A) gar die KO, so
steht dem Gegner keine KO-Probe zu.}

{\sffamily\color{black}
\item[Besonderes]: Kann mit einem
\testStyleSF{Knochenbrecher} kombiniert werden, die Ansage kann
zur TP-Steigerung oder als KO-Erschwernis eingesetzt werden.}

\section[Schwinger (Autotreffer / Raufen{}- oder
Ringen{}-Aktion)]{Schwinger
\textstyleEndnoteSymbol{(}\textstyleEndnoteSymbol{Autotreffer /
}\textstyleEndnoteSymbol{Raufen}\textstyleEndnoteSymbol{{}-
}\textstyleEndnoteSymbol{oder
Ringen}\textstyleEndnoteSymbol{{}-Aktion}\textstyleEndnoteSymbol{)}}
{\sffamily\color{black}
\begin{description}\item[Voraussetzung]: BE\textrm{ ${\leq}$ }4}

{\sffamily\color{black}
\item[Probe]: AT+Ansage (+WS)}

{\sffamily\color{black}
\item[Wirkung]: Dies ist das waffenlose Manöver der
\testStyleSF{Finte}: Der Angreifer täuscht einen Schlag auf
eine bestimmte Körperzone an, um dann doch an einer anderen Stelle zu
treffen. Die Attacke erschwert um eine Ansage ergibt einen
Qualitätsbonus in Höhe der Ansage, bei Kenntnis der SF Schwinger sogar
einen Qualitätsbonus in Höhe der doppelten Ansage.}

{\sffamily\color{black}
\item[Besonderes]: Kann mit einem
\testStyleSF{Knochenbrecher} kombiniert werden.}

\section[Schwitzkasten (kein Autotreffer /
Ringen{}-Aktion)]{Schwitzkasten
\textstyleEndnoteSymbol{(}\textstyleEndnoteSymbol{kein Autotreffer /
}\textstyleEndnoteSymbol{Ringen}\textstyleEndnoteSymbol{{}-Aktion}\textstyleEndnoteSymbol{)}}
{\sffamily\color{black}
\begin{description}\item[Voraussetzung]: SF
\testStyleSF{Schwitzkasten}}

{\sffamily\color{black}
\item[Probe]: AT}

{\sffamily\color{black}
\item[Wirkung]: Kann der Verteidiger nicht parieren,
befindet er sich im Schwitzkasten, aus dem er sich nur mit einer
Ringen-PA befreien kann. Die folgenden Ringen-AT des Angreifers sind um
1, 2, 3, etc. Punkte erleichtert und richten 1W6+1, 1W6+2, 1W6+3, etc.
Punkte AU-Verlust an (aber keinen echten Schaden). Die Paraden des
Verteidigers sind um 1, 2, 3, etc. Punkte erschwert.}

{\sffamily\color{black}
\item[Besonderes]: Ein
\testStyleSF{Schlangenmensch} erhält 5 Punkte Bonus auf seine
PA wenn er sich aus dem Schwitzkasten befreien will.}

\section[Sprungtritt (kein Autotreffer / Raufen,
speziell)]{Sprungtritt
\textstyleEndnoteSymbol{(}\textstyleEndnoteSymbol{kein Autotreffer /
Raufen, speziell}\textstyleEndnoteSymbol{)}}
{\sffamily\color{black}
\begin{description}\item[Voraussetzung]: SF
\testStyleSF{Sprungtritt}, wenigstens 4 Schritt Anlauf,
GS\textrm{ ${\geq}$ 4}}

{\sffamily\color{black}
\item[Probe]: AT+4 (+WS)}

{\sffamily\color{black}
\item[Wirkung]: Dies ist das waffenlose Manöver des
\testStyleSF{Sturmangriffs}: Ein Angriff mit Anlauf, der durch
den Schwung des Kämpfers mehr Schaden verursachen soll. Die halbe
Geschwindigkeit wird als TP-Bonus zum Schaden addiert. Dazu zusätzlich
4 Punkte und eine allfällige Ansage. Misslingt das Manöver, so entfällt
die Reaktion des Angreifers und der Verteidiger kann einen
Passierschlag ausführen.}

{\sffamily\color{black}
\item[Besonderes]: Kann mit einem
\testStyleSF{Nieder}\testStyleSF{schlagen/-werfen} und
einem \testStyleSF{Knochenbrecher} kombiniert werden. Bei
gelungener Parade erleidet der Angreifer den vollen Waffenschaden.}

\section[Unterlaufen (kein Autotreffer / Raufen{}- oder
Ringen{}-Aktion)]{Unterlaufen
\textstyleEndnoteSymbol{(}\textstyleEndnoteSymbol{kein Autotreffer /
}\textstyleEndnoteSymbol{Raufen}\textstyleEndnoteSymbol{{}-}\textstyleEndnoteSymbol{
oder
Ringen}\textstyleEndnoteSymbol{{}-Aktion}\textstyleEndnoteSymbol{)}}
{\sffamily\color{black}
\begin{description}\item[Voraussetzung]: Distanzklasse der Waffe des
Verteidigers ist länger als Handgemenge}

{\sffamily\color{black}
\item[Probe]: AT+4}

{\sffamily\color{black}
\item[Wirkung]: Diese Verzweiflungstat erlaubt dem
Angreifer eine kurze Verschnaufpause, bis der Verteidiger ihn wieder
von sich weggestossen hat. Der Angreifer nützt seine kurze
Distanzklasse aus und unterläuft den Gegner. Die nächste Aktion des
Gegners entfällt. Dieser Angriff verursacht kein Schaden.}
