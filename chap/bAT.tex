% -*-mode: Latex-*-
% !TEX root = kampf.tex
% authors: simon maurer
%
% file: bAT.tex
% contents: Attacken-Manöver des bewaffneten Kampfes
% Sccs-Id: %W% %G%

%==============================================================================
\section{Bewaffnete AT-Manöver}
\label{chap.bAT}

%------------------------------------------------------------------
\subsection{Angriff mit dem Schild \textStyleAT{(Autotreffer / Aktion)}}
\label{chap.bAT.schild}
\begin{description}
    \item[Voraussetzung]: SF \textStyleSF{Schildkampf I}
    \item[Probe]: Raufen-AT, erschwert um 3 Punkte (nicht erschwert bei
        Kenntnis von \textStyleSF{Schildkampf II}) und den AT-WM des Schildes
        (+WS)
    \item[Wirkung]: Wenn nichts anderes Angegeben ist, verursacht ein Schild
        1W+1 TP(A) mit TP/KK 13/3. Ein mit Dornen ausgerüsteter Schild
        verursacht echte TP.
    \item[PA für den Gegner]: Ein Angriff mit Schild (Ausnahmen: Buckler und
        Bock) kann mit Dolchen, Fechtwaffen und Kettenwaffen nicht pariert
        werden. Wird der Angriff pariert entsteht auf keiner Seite Schaden.
    \item[Besonderes]: Kann mit einem \textStyleSF{Wuchtschlag}, einem
        \textStyleSF{Sturmangriff} und einem \textStyleSF{Niederwerfen}
        kombiniert werden.
\end{description}

%------------------------------------------------------------------
\subsection{Auf Distanz halten \textStyleAT{(kein Autotreffer / Aktion)}}
\label{chap.bAT.distanz}
\begin{description}
    \item[Voraussetzung]: Distanzklasse der Waffe des
        Angreifers ist länger als jene des Verteidigers
    \item[Probe]: AT (+F)
    \item[Wirkung]: Der Angreifer nützt die Länge seiner Waffe aus und bringt
        sich so in Position, dass die die nächste Aktion des Gegners entfällt.
        Dieser Angriff verursacht kein Schaden.
    \item[Besonderes]: Kann mit einer \textStyleSF{Finte} kombiniert werden.
\end{description}

%------------------------------------------------------------------
\subsection{Ausfall \textStyleAT{(Autotreffer / Aktion)}}
\label{chap.bAT.ausfall}
\begin{description}
    \item[Voraussetzung]: SF \textStyleSF{Ausfall}, BE \textrm{${\leq}$} 4,
        ausreichende Bewegungsfreiheit
    \item[Probe]: Einleitende AT+4, jede weitere AT normal.  Trägt der
        Ausfallende ein Schild ist jede AT um (weitere) 2 Punkte erschwert (+F)
        (+WS)
    \item[Wirkung]: Der Angreifer greift mit einer Folge von Attacken an und
        lässt seinem Gegner keine Zeit für einen Gegenangriff, sondern drängt
        ihn mit jedem Schlag zurück. Der Ausfall endet, wenn
        \textStyleStrongEmphasis{eine AT eine negative Qualität} aufweist, wenn
        dem Gegner \textStyleStrongEmphasis{eine glückliche PA} gelingt,
        \textStyleStrongEmphasis{KO*AT geschlagen} wurden, der Gegner
        \textStyleStrongEmphasis{stehen bleibt} (MU+Qualität, bei misslingen
        geht Ausfall weiter, in jedem Fall PA+4),
        \textStyleStrongEmphasis{meisterlich pariert} (mit einer minimalen
        Ansage von 4) oder nicht mehr weiter zurückweichen kann (in diesem Fall
        PA+4 für den Verteidiger).
    \item[Besonderes]: Während des Ausfalls können der \textStyleSF{Wuchtschag}
        und die \textStyleSF{Finte} eingesetzt werden (maximale Ansage: +4).
        Als abschliessende AT ist auch ein \textStyleSF{Angriff zum
        Niederwerfen}, ein \textStyleSF{Hammerschlag}, ein
        \textStyleSF{Gezielter Stich} oder ein \textStyleSF{Todesstoss}
        möglich.
\end{description}

%------------------------------------------------------------------
\subsection{Befreiungsschlag \textStyleAT{(Autotreffer / Aktion+Reaktion)}}
\label{chap.bAT.befreiungsschlag}
\begin{description}
    \item[Voraussetzung]: SF \textStyleSF{Befreiungsschlag}
    \item[Probe]: AT+(4 pro Gegner) (+F) (+WS)
    \item[Wirkung]: Ein Rundumschlag, mit dem mehrere Gegner gleichzeitig
        angegriffen werden. Jeder Gegner dem die PA misslingt erleidet einen
        Treffer, und verliert die folgende AT. Können die Gegner aus
        irgendwelchen Gründen nicht zurückweichen, erleiden sie 1W6 Punkte
        zusätzlichen Schaden.
    \item[Besonderes]: Kann mit einem \textStyleSF{Wuchtschlag} und einer
        \textStyleSF{Finte} kombiniert werden.
\end{description}

%------------------------------------------------------------------
\subsection{Betäubungsschlag \textStyleAT{(Autotreffer / Aktion)}}
\label{chap.bAT.betaeubungsschlag}
\begin{description}
    \item[Voraussetzung]: SF \textStyleSF{Betäubungsschlag}, Waffe mit stumpfer
        Seite
    \item[Probe]: Kampfstäbe, stumpfe Hiebwaffen AT+2, stumpfe Seiten anderer
        Hiebwaffen AT+4, andere mögliche Waffen AT+8, mit Knauf Raufen-AT+4
        (+WS)
    \item[Wirkung]: Ein starker Schlag, der den Gegner bewusstlos schlagen
        soll, ohne diesen schwer zu verletzen. Der Schlag richtet keine
        richtigen TP sondern nur TP(A) an. Übersteigen die TP(A) die
        Wundschwelle des Gegners, muss dieser eine KO-Probe ablegen. Bei
        Misslingen, fällt das Opfer für 1W6 SR in Ohnmacht. Übersteigen die
        TP(A) gar die KO, so steht dem Gegner keine KO-Probe zu.
    \item[Besonderes]: Kann mit einem \textStyleSF{Wuchtschlag} kombiniert
        werden, die Ansage kann zur TP-Steigerung oder als KO-Erschwernis
        eingesetzt werden.
\end{description}

%------------------------------------------------------------------
\subsection{Doppelangriff \textStyleAT{(Autotreffer / Aktion)}}
\label{chap.bAT.doppelangriff}
\begin{description}
    \item[Voraussetzung]: SF \textStyleSF{Doppelangriff}, zwei Einhand-Waffen
        die in der gleichen DK verwendbar sind, keine Kettenwaffen und Speere
    \item[Probe]: AT+2 pro Hand, bei nicht identischen Waffen Zweithand
        zusätzlich +4, Zweithand zusätzlich +3 ohne Beidhändiger Kampf II
    \item[Wirkung]: Mit beiden Waffen wird gleichzeitig zugeschlagen. Es werden
        zwei Paraden benötigt (Defensiver Kampfstil, Klingenwand, PA und
        Gezieltes Ausweichen, PA und zusätzliche PA von Zweitwaffe).
\end{description}

%------------------------------------------------------------------
\subsection{Entwaffnen \textStyleAT{(kein Autotreffer / Aktion)}}
\label{chap.bAT.entwaffnen}
\begin{description}
    \item[Voraussetzung]: SF \textStyleSF{Entwaffnen}
    \item[Probe]: AT+8 (+F)
    \item[Wirkung]: Dieser schnelle Angriff zielt nicht auf eine Verletzung ab,
        sondern soll den Gegner entwaffnen. Misslingt die PA des Verteidigers,
        muss er eine KK-Probe erschwert um 8 (um 10 bei Kenntnis der SF
        \textStyleSF{Meisterliches Entwaffnen}) bestehen, sonst verliert er die
        Waffe. Bei diesem Angriff entsteht beim Getroffenen kein Schaden.
    \item[Besonderes]: Mit Kampfstäben ist dieses Manöver um 2 Punkte, mit
        Kettenstäben und Zweililien um 4 Punkte erleichtert.  Entwaffnen ist
        üblicherweise nur gegen einhändig geführte Waffen möglich. Wer
        allerdings die SF \textStyleSF{Meisterliches Entwaffnen} beherrscht,
        kann auch Gegner mit Zweihandwaffen entwaffnen.  Das Vorgehen
        entspricht dem, was auch bei Einhandwaffen gilt.
    \item[Besonderes]: Kann mit einer \textStyleSF{Finte} kombiniert werden,
        die Ansage kann als PA-Erschwernis oder KK-Erschwernis eingesetzt
        werden.
\end{description}

%------------------------------------------------------------------
\subsection{Festnageln \textStyleAT{(Autotreffer
\label{chap.bAT.festnageln}
/ Aktion+Reaktion)}}
\begin{description}
    \item[Voraussetzung]: SF \textStyleSF{Festnageln}, Stangenwaffe oder Speer
        mit gerader Klinge und einem Knebel, der ein zu tiefes eindringen der
        Klinge verhindert.
    \item[Probe, Wirkung]: Siehe S. 62 WdS.
\end{description}

%------------------------------------------------------------------
\subsection{Finte \textStyleAT{(Autotreffer / Aktion)}}
\label{chap.bAT.finte}
\begin{description}
    \item[Voraussetzung]: BE\textrm{ ${\leq}$ }4
    \item[Probe]: AT+Ansage (+WS)
    \item[Wirkung]: Der Angreifer täuscht einen Schlag auf eine bestimmte
        Körperzone an, um dann die Waffe umzulenken und an einer anderen Stelle
        zu treffen. Die Attacke erschwert um eine Ansage ergibt einen
        Qualitätsbonus in Höhe der Ansage, bei Kenntnis der SF Finte sogar
        einen Qualitätsbonus in Höhe der doppelten Ansage.
    \item[Besonderes]: Kann mit einem \textStyleSF{Wuchtschlag} kombiniert
        werden.
\end{description}

%------------------------------------------------------------------
\subsection{Gezielter Stich \textStyleAT{(Autotreffer / Aktion)}}
\label{chap.bAT.stich}
\begin{description}
    \item[Voraussetzung]: SF \textStyleSF{Gezielter Stich}
    \item[Probe]: AT+4 + halber gegnerischer RS (+F)
    \item[Wirkung]: Diese Attacke richtet sich gezielt gegen eine verwundbare,
        ungerüstete Stelle des Gegners. Erstens wird die Rüstung des Gegners
        ignoriert (auch natürlicher RS), zweitens ist seine Wundschwelle bei
        dieser Art von Angriff um 2 Punkte gesenkt und drittens erleidet er
        automatisch eine Wunde.
    \item[Besonderes]: Kann mit einer \textStyleSF{Finte} kombiniert werden.
\end{description}

%------------------------------------------------------------------
\subsection{Hammerschlag \textStyleAT{(Autotreffer / Aktion+alle
\label{chap.bAT.hammerschlag}
Reaktionen)}}
\begin{description}
    \item[Voraussetzung]: SF \textStyleSF{Hammerschlag}
    \item[Probe]: AT+8 (+WS) (+F)
    \item[Wirkung]: Mit diesem riskanten und mit aller Kraft geführten Schlag
        versucht der Kämpfer, den Kampf möglichst mit einem einzelnen Schlag zu
        beenden. Die TP werden verdreifacht (das beinhaltet auch TP/KK und
        allfällige Ansage aus Wuchtschlag).
    \item[Besonderes]: Kann mit einem \textStyleSF{Wuchtschlag} und einer
        \textStyleSF{Finte} (wenn die Waffe eine Finte erlaubt) kombiniert
        werden.
\end{description}

%------------------------------------------------------------------
\subsection{Klingensturm \textStyleAT{(Autotreffer / Aktion)}}
\label{chap.bAT.klingensturm}
\begin{description}
    \item[Voraussetzung]: SF \textStyleSF{Klingensturm}, Kämpfer führt keinen
        Schild, BE\textrm{ ${\leq}$ }4
    \item[Probe]: Zwei AT gegen zwei unterschiedliche Gegner, jeweils auf
        (AT/2)+2.\newline
        Bei Kenntnis der SF \textStyleSF{Kampfgespür} können AT+4 Punkte frei
        auf zwei AT gegen zwei Gegner verteilt werden, wobei der Mindestwert
        einer AT 6 beträgt.\newline
        Bei Kenntnis der SF \textStyleSF{Klingentänzer} können AT+6 Punkte frei
        auf drei AT gegen drei Gegner verteilt werden, wobei der Mindestwert
        einer AT 6 beträgt.
    \item[Wirkung]: Mit dieser Attacke kann eine Kämpferin zwei Gegner in
        direkter Folge angreifen, also mit einem einzigen Angriffsmanöver.
\end{description}

%------------------------------------------------------------------
\subsection{Niederwerfen \textStyleAT{(Autotreffer / Aktion)}}
\label{chap.bAT.niederwerfen}
\begin{description}
    \item[Voraussetzung]: SF \textStyleSF{Niederwerfen}
    \item[Probe]: AT+4(+Ansage) (+WS)
    \item[Wirkung]: Der Schlag verursacht Schaden wie üblich.  Zusätzlich muss
        der Gegner eine KK-Probe, erschwert um eine allfällige Ansage,
        Erleichterungen für \textStyleSF{Standfest} (-1), \textStyleSF{Balance}
        (-2), \textStyleSF{Herausragende Balance} (-4) bestehen um auf den
        Beinen zu bleiben.
    \item[Besonderes]: Kann mit einem \textStyleSF{Sturmangriff} und einem
        \textStyleSF{Wuchtschlag} kombiniert werden.
\end{description}

%------------------------------------------------------------------
\subsection{Passierschlag \textStyleAT{(kein Autotreffer / Freie Raufen oder
\label{chap.bAT.passierschlag}
Ringen-Aktion)}}
\begin{description}
    \item[Voraussetzung]: Ein Kämpfer durchschreitet den eigenen
        Kontrollbereich oder gibt sich durch eine Aktion eine Blösse
    \item[Probe]: AT+4
    \item[Wirkung]: Als Freie Aktion kann einem unvorsichtigen Kämpfer, der den
        eigenen Kontrollbereich durchschritten oder sich durch ein Manöver in
        eine unglückliche Lage begeben hat (Sturmangriff) ein ungezielter
        Schlag versetzt werden (keine Manöver).  Besitzt das Opfer die SF
        \textStyleSF{Aufmerksamkeit} ist der Schlag um weitere 4 Punkte
        erschwert, bei der SF \textStyleSF{Kampfgespür} gar um 6 Punkte (4 aus
        \textStyleSF{Aufmerksamkeit} und 2 aus \textStyleSF{Kampfgespür}).
\end{description}

%------------------------------------------------------------------
\subsection{Schildspalter \textStyleAT{(kein Autotreffer / Aktion)}}
\label{chap.bAT.schildspalter}
\begin{description}
    \item[Voraussetzung]: SF \textStyleSF{Schildspalter}
    \item[Probe]: AT+8(+Ansage), erleichtert um PA-WM des Schildes
    \item[Wirkung]: Mit einer geeigneten Waffe ist es möglich, den Schild einer
        Gegnerin zu spalten oder so zu verbeulen, dass er jeden Nutzen
        verliert. Misslingt die Parade des Verteidigers, muss er einen
        Bruchfaktortest erschwert um eine allfällige Ansage würfeln.
\end{description}

%------------------------------------------------------------------
\subsection{Stumpfer Schlag \textStyleAT{(Autotreffer / Aktion)}}
\label{chap.bAT.stumpf}
\begin{description}
    \item[Voraussetzung]: keine
    \item[Probe]: Kampfstäbe, stumpfe Hiebwaffen AT+2, stumpfe Seiten anderer
        Hiebwaffen AT+4, andere mögliche Waffen AT+8, mit Knauf Raufen-AT+4.
        Bei Kenntnis der SF \textStyleSF{Betäubungsschlag}, werden die
        genannten AT-Zuschläge halbiert.
    \item[Wirkung]: Der Angreifer schlägt gebremst oder mit der flachen Seite
        der Waffe zu, um möglichst wenig realen Schaden, sondern nur
        Betäubungsschaden zu verursachen. Solche Schläge richten TP(A) statt TP
        an.
\end{description}

%------------------------------------------------------------------
\subsection{Sturmangriff \textStyleAT{(kein Autotreffer / speziell)}}
\label{chap.bAT.sturmangriff}
\begin{description}
    \item[Voraussetzung]: SF \textStyleSF{Sturmangriff}, wenigstens 4 Schritt
        Anlauf, GS\textrm{ ${\geq}$ 4}
    \item[Probe]: AT+4 (+WS)
    \item[Wirkung]: Ein Angriff mit Anlauf, der durch den Schwung des Kämpfers
        mehr Schaden verursachen soll. Die halbe Geschwindigkeit wird als
        TP-Bonus zum Schaden addiert. Dazu zusätzlich 4 Punkte und eine
        allfällige Ansage. Misslingt das Manöver, so entfällt die Reaktion des
        Angreifers und der Verteidiger kann einen Passierschlag ausführen.
    \item[Besonderes]: Kann mit einem \textStyleSF{Niederwerfen} und einem
        \textStyleSF{Wuchtschlag} kombiniert werden.
\end{description}

%------------------------------------------------------------------
\subsection{Tod von Links \textStyleAT{(Autotreffer / Aktion)}}
\label{chap.bAT.tod}
\begin{description}
    \item[Voraussetzung]: SF \textStyleSF{Tod von Links}, BE\textrm{ ${\leq}$
        }4
    \item[Probe]: AT mit dem AT-Wert der Parierwaffe, zusätzlich +3 ohne
        \textStyleSF{Beidhändiger Kampf II} (+F)
    \item[Wirkung]: Eine zusätzliche Attacke mit der falschen Hand (nicht
        kumulierbar mit einer Zusatzaktion aus \textStyleSF{Beidhändiger Kampf
        II}), die mit der Parierwaffe ausgeführt werden muss.
    \item[Besonderes]: Kann mit einer \textStyleSF{Finte} kombiniert werden.
\end{description}

%------------------------------------------------------------------
\subsection{Todesstoss \textStyleAT{(Autotreffer / Aktion+alle
\label{chap.bAT.todesstoss}
Reaktionen)}}
\begin{description}
    \item[Voraussetzung]: SF \textStyleSF{Todesstoss}
    \item[Probe]: AT+8 + halber gegnerischer RS (+F) (+WS)
    \item[Wirkung]: Eine riskante Version des gezielten Stichs, mit dem der
        Gegner mit einem einzigen Stich kampfunfähig gemacht werden soll. Der
        RS wird ignoriert (nicht aber natürlicher RS).  Um die SP zu ermitteln
        wird der Schaden (Waffen-Schaden inklusive KK-Bonus) verdoppelt. Die
        Wundschwelle des Verteidigers ist um 2 reduziert und es werden
        automatisch zwei Wunden verursacht.
    \item[Besonderes]: Kann mit einer \textStyleSF{Finte} und einem
        \textStyleSF{Wuchtschlag} (selbst wenn die Waffe keinen
        \textStyleSF{Wuchtschlag} erlauben würde) kombiniert werden.
\end{description}

%------------------------------------------------------------------
\subsection{Umreissen \textStyleAT{(kein Autotreffer / Aktion)}}
\label{chap.bAT.umreissen}
\begin{description}
    \item[Voraussetzung]: SF \textStyleSF{Umreissen}, Kämpfer führt weder
        Schild noch Parierwaffe
    \item[Probe]: AT+8 (+F)
    \item[Wirkung]: Bei dieser Variante der Finte soll der Gegner durch
        geschickte Platzierung des Treffers zu Boden gezwungen werden. Kann
        nicht Pariert werden, muss das Opfer eine um die TP des Angriffs
        erschwerte GE Probe bestehen um auf den Beinen zu bleiben
        (Erleichterungen für \textStyleSF{Standfest} (-2),
        \textStyleSF{Balance} (-4) und \textStyleSF{Herausragende Balance}
        (-8)). Ein Treffer erzeugt keinen Schaden.
    \item[PA für den Gegner]: Der Angriff kann nur mit einer PA+8, einer
        Raufen- oder Ringen-PA mit Beinarbeit oder mit einem Ausweichen pariert
        werden.
    \item[Besonderes]: Kann mit einer \textStyleSF{Finte} kombiniert werden.
\end{description}

%------------------------------------------------------------------
\subsection{Unterlaufen \textStyleAT{(kein Autotreffer / Aktion)}}
\label{chap.bAT.unterlaufen}
\begin{description}
    \item[Voraussetzung]: Distanzklasse der Waffe des Angreifers ist kürzer als
        jene des Verteidigers
    \item[Probe]: AT+4 (+F)
    \item[Wirkung]: Diese Verzweiflungstat erlaubt dem Angreifer eine kurze
        Verschnaufpause, bis der Verteidiger ihn wieder von sich weggestossen
        hat. Der Angreifer nützt die Kürze seiner Waffe aus und unterläuft den
        Gegner. Die nächste Aktion des Gegners entfällt.  Dieser Angriff
        verursacht kein Schaden.
    \item[Besonderes]: Kann mit einer \textStyleSF{Finte} kombiniert werden.
\end{description}

%------------------------------------------------------------------
\subsection{Wuchtschlag \textStyleAT{(Autotreffer / Aktion)}}
\label{chap.bAT.wuchtschlag}
\begin{description}
    \item[Voraussetzung]: keine
    \item[Probe]: AT+Ansage (+F)
    \item[Wirkung]: Der Kämpfer führt einen besonders kraftvollen Schlag aus.
        Die TP werden um die hälfte der Ansage erhöht.  Bei Kenntnis der SF
        \textStyleSF{Wuchtschlag} werden die TP gar um die ganze Ansage erhöht.
    \item[Besonderes]: Kann mit einer \textStyleSF{Finte} kombiniert werden.
\end{description}
