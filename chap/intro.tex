% -*-mode: Latex-*-
% !TEX root = kampf.tex
% authors: Kaspar Manz, Simon Maurer
%
% file: chap/intro.tex
% contents: introduction to the paper
% Sccs-Id: %W% %G%

\chapter{Einleitung}
Wer erinnert sich nicht an Quentin Tarantinos "Kill Bill"? Die Kämpfe sind – nicht wie in anderen Filmen – schnell und dramatisch, ihre Folgen drastisch und blutig. QVAT verhält sich ähnlich zum Original-Kampfsystem: die Kämpfe werden schneller, härter und insgesamt cineastischer.

% Damit birgt es aber auch eine Gefahr: die Bandbreite der möglichen Trefferpunkte erhöht sich unter QVAT beträchtlich, der Kampfausgang ist weit weniger berechenbar.
% Nicht nur Meisterpersonen sterben schneller – auch Helden sind dieser Gefahr ausgesetzt.
% Unter unglücklichen Umständen – ein einziger glücklicher Treffer des Gegners kann schon ausreichen – holt sich damit Golgari die Seele des Helden schneller als dessen Spieler und Meister lieb ist.

% Unter QVAT empfiehlt es sich deshalb nicht, Kämpfe gegen Meisterfiguren offen auszuwürfeln, noch sollte der Meister seinen Würfelresultaten sklavisch folgen müssen.
% Vielmehr sollte sich der Meister angewöhnen, die Kämpfe dramatisch auszugestalten und unabhängig von seinen eigenen Würfelresultaten zu lenken.
% Dies gilt vor allem auch für Überzahlsituationen, die durch QVAT um einiges an Gefährlichkeit zugelegt haben.

Die Zielsetzung bei der Entwicklung des QVAT war, ein System zu haben, das den Spielablauf verschlankt und beschleunigt, ohne dabei die grundlegenden Werte der Helden anpassen zu müssen.
Zudem sollte die paradoxe Situation verhindert werden, dass selbst geübte Kämpfer zwischendurch Phasen haben, in denen sie scheinbar ihre Waffe nicht unter Kontrolle haben und nicht in der Lage sind, ihren Gegner zu treffen.
QVAT geht davon aus, dass Kämpfer prinzipiell immer in der Lage sind, in der Richtung ihres Gegners zu schlagen – dies jedoch nicht immer in der selben Qualität schaffen.

Einige regeltechnische Simulationen, die im originalen Kampfsystem für mehr Realismus sorgen sollten, wurden der Einfachkeit halber in QVAT beiseite gelassen: dazu gehören etwa die Distanzklassen oder die Initiativreihenfolge.
Statt all diese Dinge zu reglementieren, wurde beschlossen, diese dem gesunden Menschenverstand der Spieler und des Meisters zu überlassen.
Es sollte sich von selbst verstehen, dass zum Beispiel ein Meuchelmörder mit seinem Dolch gegen einen Speerkämpfer von vorne nicht all zu grosse Chancen hat, einen Treffer zu landen...

% \section{Vorwort}

% \section{Wie man dieses Dokument benutzt}

