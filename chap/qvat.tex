% -*-mode: Latex-*-
% !TEX root = kampf.tex
% authors: simon maurer
%
% file: qvat.tex
% contents: Kurze Beschreibung des QVAT Systems
% Sccs-Id: %W% %G%

%==============================================================================
\chapter{QVAT}
Nach QVAT gelingt jede Attacke, es sei denn man würfelt einen Patzer.
Jede Attacke hat eine Qualität.
Die Qualität berechnet sich aus der Differenz von AT-Wurf und AT-Wert geteilt durch zwei.
Liegt der AT-Wurf über dem AT-Wert, liegt eine negative Qualität vor, ansonsten eine positive.
Eine negative Qualität zählt als Parade-Erleichterung für den Verteidiger und wird vom Schaden abgezogen (Ausnahme: ein Kämpfer mit der SF Klingentänzer muss eine negative Qualität nicht vom Schaden abziehen), falls die Parade trotzdem misslingt.
Eine positive Qualität zählt als Parade-Erschwernis für den Verteidiger, modifiziert aber den Schaden nicht.

Eine Parade gelingt, wenn nicht höher als der durch die Qualität des Angreifers modifizierte PA-Wert gewürfelt wird.
