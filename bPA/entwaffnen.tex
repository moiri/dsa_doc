\section{Entwaffnen}
\label{bPA.entwaffnen}
\begin{description}
    \item[Voraussetzung:]
        SF \textStyleSF{\nameref{sf.entwaffnen}}
    \item[Probe:]
        PA+8
    \item[Wirkung:]
        Gelingt die PA des Verteidigers, muss der Angreifer eine KK-Probe erschwert um 8 (um 10 bei Kenntnis der SF \textStyleSF{\nameref{sf.meisterliches_entwaffnen}}) bestehen, sonst verliert er die Waffe.
    \item[Besonderes:]
        Mit Bock, Hakendolch, Linkhand und ev. Drachenklaue ist dieses Manöver um 2 Punkte erleichtert (benötigt Kenntnis der SF \textStyleSF{\nameref{sf.parierwaffen} I}).
        Entwaffnen ist üblicherweise nur gegen einhändig geführte Waffen möglich.
        Wer allerdings die SF \textStyleSF{\nameref{sf.meisterliches_entwaffnen}} beherrscht, kann auch Gegner mit Zweihandwaffen entwaffnen.
        Das Vorgehen entspricht dem, was auch bei Einhandwaffen gilt.
    \item[Besonderes:]
        Eine zusätzliche Ansage kann als KK-Erschwernis eingesetzt werden.
\end{description}
