\section{Gegenhalten}
\label{bPA.gegenhalten}
\begin{description}
    \item[Voraussetzung:]
        SF \textStyleSF{\nameref{sf.gegenhalten}}
    \item[Probe:]
        AT +4
    \item[Wirkung:]
        Derjenige mit der besseren Qualität macht vollen Schaden, derjenigen mit der schlechteren nur den halben.
        Bei gleicher Güte gewinnt der ursprüngliche Angreifer.
        Eine negative Qualität wird wie üblich vom Schaden abgezogen (gegebenenfalls zuerst halbieren).
\end{description}
