\section{Unterlaufen}
\label{bPA.unterlaufen}
\begin{description}
    \item[Voraussetzung:]
        Distanzklasse der Waffe des Verteidigers ist kürzer als jene des Angreifers
    \item[Probe:]
        PA+8
    \item[Wirkung:]
        Bei dieser waghalsigen Aktion nützt der Verteidiger die Kürze seiner Waffe aus und unterläuft den Gegner.
        Die nächste Reaktion des Gegners entfällt.
        Kennt der Gegner die SF \textStyleSF{\nameref{sf.halbschwert}}, kann er trotz dem Unterlaufen dennoch parieren (ohne Einsatz eines anderen Manövers).
\end{description}
