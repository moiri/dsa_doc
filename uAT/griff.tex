\section{Griff}
\label{uAT.griff}
\textStyleAT{(kein Autotreffer / Ringen-Aktion)}
\begin{description}
    \item[Voraussetzung:]
        SF \textStyleSF{\nameref{sf.griff}}
    \item[Probe:]
        AT +Ansage (+F)
    \item[Wirkung:]
        Ein Griff soll den Gegner kampfunfähig machen und zielt nicht darauf ab, Schaden zu verursachen.
        Kann der Angriff nicht Pariert werden, ist jede folgende Aktion des Gegriffenen um die doppelte Ansage erschwert.
        Der Griff hält an, bis der Greifende freiwillig loslässt oder der Gegriffene sich befreien kann:
        Der Gegriffene kann versuchen, den Griff mit einer (nicht erschwerten) Raufen oder Ringen-AT abzuschütteln, die wiederum vom Greifenden mit einer normalen Ringen-PA pariert werden kann.
    \item[Besonderes:]
        Kann mit einem \textStyleSF{\nameref{sf.schwinger}} kombiniert werden.
\end{description}
