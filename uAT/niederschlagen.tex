\section{Niederschlagen/-werfen}
\label{uAT.niederschlagen}
\textStyleAT{(Autotreffer / Raufen- oder Ringen-Aktion)}
\begin{description}
    \item[Voraussetzung:]
        SF \textStyleSF{\nameref{sf.knochenbrecher}}
    \item[Probe:]
        AT+4(+Ansage) (+WS)
    \item[Wirkung:]
        Dies ist das waffenlose Manöver des \textStyleM{\nameref{bAT.niederwerfen}s}.
        Das Manöver verursacht Schaden wie üblich. Zusätzlich muss der Gegner eine KK-Probe, erschwert um eine allfällige Ansage, Erleichterungen für \textStyleSF{Standfest} (-1), \textStyleSF{Balance} (-2), \textStyleSF{Herausragende Balance} (-4) bestehen um auf den Beinen zu bleiben.
        Beherrscht der Angreifer die SF \textStyleSF{\nameref{sf.wurf} (nur Ringen)} und wurde der Verteidiger zuvor mit dem Manöver \textStyleM{\nameref{uAT.griff}} gegriffen, so steht dem Verteidiger keine Probe zum Stehenbleiben zu.
    \item[Besonderes:]
        Kann mit einem \textStyleSF{\nameref{uAT.sprungtritt}} und einem \textStyleSF{\nameref{uAT.knochenbrecher}} kombiniert werden.
\end{description}

