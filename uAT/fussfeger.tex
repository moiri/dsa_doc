\section{Fussfeger}
\label{uAT.fussfeger}
\textStyleAT{(kein Autotreffer / Raufen-Aktion)}
\begin{description}
    \item[Voraussetzung:]
        SF \textStyleSF{\nameref{sf.fussfeger}}
    \item[Probe:]
        AT+8 (+F)
    \item[Wirkung:]
        Dies ist das waffenlose Manöver des \textStyleM{\nameref{bAT.umreissen}s}:
        Kann nicht Pariert werden, muss das Opfer eine um die TP(A) des Angriffs erschwerte GE Probe bestehen um auf den Beinen zu bleiben (Erleichterungen für \textStyleSF{Standfest} (-2), \textStyleSF{Balance} (-4) und \textStyleSF{Herausragende Balance} (-8)).
        Ein Treffer erzeugt keinen Schaden. 
    \item[PA für den Gegner:]
        Der Angriff kann nur mit einer Waffen-PA+8, einer Raufen- oder Ringen-PA mit \textStyleSF{\nameref{sf.beinarbeit}} oder mit einem \textStyleSF{\nameref{sf.ausweichen}} pariert werden.
    \item[Besonderes:]
        Kann mit einem \textStyleM{\nameref{uAT.schwinger}} kombiniert werden.
\end{description}

