\section{Schmetterschlag}
\label{uAT.schmetterschlag}
\textStyleAT{(Autotreffer / Raufen-Aktion)}
\begin{description}
    \item[Voraussetzung:]
        SF \textStyleSF{\nameref{sf.schmetterschlag}}
    \item[Probe:]
        AT+2
    \item[Wirkung:]
        Dies ist das waffenlose Manöver des \textStyleM{\nameref{bAT.betaeubungsschlag}s}:
        Ein starker Schlag, der den Gegner bewusstlos schlagen soll.
        Übersteigen die TP(A) die Wundschwelle des Gegners, muss dieser eine KO-Probe ablegen.
        Bei Misslingen, fällt das Opfer für 1W6 SR in Ohnmacht.
        Übersteigen die TP(A) gar die KO, so steht dem Gegner keine KO-Probe zu.
    \item[Besonderes:]
        Kann mit einem \textStyleM{\nameref{uAT.knochenbrecher}} kombiniert werden, die Ansage kann zur TP-Steigerung oder als KO-Erschwernis eingesetzt werden.
\end{description}
