\section{Umreissen}
\label{bAT.umreissen}
\textStyleAT{(kein Autotreffer / Aktion)}
\begin{description}
    \item[Voraussetzung:]
        SF \textStyleSF{\nameref{sf.umreissen}}, Kämpfer führt weder Schild noch Parierwaffe
    \item[Probe:]
        AT+8 (+F)
    \item[Wirkung:]
        Bei dieser Variante der \textStyleM{\nameref{bAT.finte}} soll der Gegner durch geschickte Platzierung des Treffers zu Boden gezwungen werden.
        Kann nicht Pariert werden, muss das Opfer eine um die TP des Angriffs erschwerte GE Probe bestehen um auf den Beinen zu bleiben (Erleichterungen für \textStyleSF{Standfest} (-2), \textStyleVT{Balance} (-4) und \textStyleVT{Herausragende Balance} (-8)).
        Ein Treffer erzeugt keinen Schaden.
    \item[PA für den Gegner:]
        Der Angriff kann nur mit einer PA+8, einer \textStyleTa{Raufen}- oder \textStyleTa{Ringen}-PA mit \textStyleSF{Beinarbeit} oder mit einem Ausweichen pariert werden.
    \item[Besonderes:]
        Kann mit einer \textStyleM{\nameref{bAT.finte}} kombiniert werden.
\end{description}
