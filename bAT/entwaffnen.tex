\section{Entwaffnen}
\label{bAT.entwaffnen}
\textStyleAT{(kein Autotreffer / Aktion)}
\begin{description}
    \item[Voraussetzung:]
        SF \textStyleSF{\nameref{sf.entwaffnen}}
    \item[Probe:]
        AT+8 (+F)
    \item[Wirkung:]
        Dieser schnelle Angriff zielt nicht auf eine Verletzung ab, sondern soll den Gegner entwaffnen.
        Misslingt die PA des Verteidigers, muss er eine KK-Probe erschwert um 8 (um 10 bei Kenntnis der SF \textStyleSF{\nameref{sf.meisterliches_entwaffnen}}) bestehen, sonst verliert er die Waffe.
        Bei diesem Angriff entsteht beim Getroffenen kein Schaden.
    \item[Besonderes:]
        Mit Kampfstäben ist dieses Manöver um 2 Punkte, mit Kettenstäben und Zweililien um 4 Punkte erleichtert.
        Entwaffnen ist üblicherweise nur gegen einhändig geführte Waffen möglich.
        Wer allerdings die SF \textStyleSF{\nameref{sf.meisterliches_entwaffnen}} beherrscht, kann auch Gegner mit Zweihandwaffen entwaffnen.
        Das Vorgehen entspricht dem, was auch bei Einhandwaffen gilt.
    \item[Besonderes:]
        Kann mit einer \textStyleM{\nameref{bAT.finte}} kombiniert werden, die Ansage kann als PA-Erschwernis oder KK-Erschwernis eingesetzt werden.
\end{description}
