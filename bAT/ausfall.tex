\section{Ausfall}
\label{bAT.ausfall}
\textStyleAT{(Autotreffer / Aktion)}
\begin{description}
    \item[Voraussetzung:]
        SF \textStyleSF{\nameref{sf.ausfall}}, BE \textrm{${\leq}$} 4, ausreichende Bewegungsfreiheit
    \item[Probe:]
        Einleitende AT+4, jede weitere AT normal.
        Trägt der Ausfallende ein Schild ist jede AT um (weitere) 2 Punkte erschwert (+F) (+WS)
    \item[Wirkung:]
        Der Angreifer greift mit einer Folge von Attacken an und lässt seinem Gegner keine Zeit für einen Gegenangriff, sondern drängt ihn mit jedem Schlag zurück.
        Der Ausfall endet, wenn \textStyleStrongEmphasis{eine AT eine negative Qualität} aufweist, wenn dem Gegner \textStyleStrongEmphasis{eine glückliche PA} gelingt, \textStyleStrongEmphasis{KO*AT geschlagen} wurden, der Gegner \textStyleStrongEmphasis{stehen bleibt} (MU+Qualität, bei misslingen geht Ausfall weiter, in jedem Fall PA+4), \textStyleStrongEmphasis{meisterlich pariert} (mit einer minimalen Ansage von 4) oder nicht mehr weiter zurückweichen kann (in diesem Fall PA+4 für den Verteidiger).
    \item[Besonderes:]
        Während des Ausfalls können der \textStyleM{\nameref{bAT.wuchtschlag}} und die \textStyleM{\nameref{sf.finte}} eingesetzt werden (maximale Ansage: +4).
        Als abschliessende AT ist auch ein \textStyleM{\nameref{bAT.niederwerfen}}, ein \textStyleM{\nameref{bAT.hammerschlag}}, ein \textStyleM{\nameref{bAT.gezielter_stich}} oder ein \textStyleM{\nameref{bAT.todesstoss}} möglich.
\end{description}
