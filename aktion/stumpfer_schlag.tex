\section{Stumpfer Schlag}
\label{aktion.stumpfer_schlag}
\textStyleAT{(Autotreffer / Aktion)}
\begin{description}
    \item[Voraussetzung:]
        keine
    \item[Probe:]
        Kampfstäbe, stumpfe Hiebwaffen AT+2, stumpfe Seiten anderer Hiebwaffen AT+4, andere mögliche Waffen AT+8, mit Knauf Raufen-AT+4.
        Bei Kenntnis der SF \textStyleSF{\nameref{sf.betaeubungsschlag}}, werden die genannten AT-Zuschläge halbiert.
    \item[Wirkung:]
        Der Angreifer schlägt gebremst oder mit der flachen Seite der Waffe zu, um möglichst wenig realen Schaden, sondern nur Betäubungsschaden zu verursachen.
        Solche Schläge richten TP(A) statt TP an.
\end{description}
