\section{Betäubungsschlag}
\label{aktion.betaeubungsschlag}
\textStyleAT{(Autotreffer / Aktion)}
\begin{description}
    \item[Voraussetzung:]
        SF \textStyleSF{\nameref{sf.betaeubungsschlag}}, Waffe mit stumpfer Seite
    \item[Probe:]
        Kampfstäbe, stumpfe Hiebwaffen AT+2, stumpfe Seiten anderer Hiebwaffen AT+4, andere mögliche Waffen AT+8, mit Knauf Raufen-AT+4 (+WS)
    \item[Wirkung:]
        Ein starker Schlag, der den Gegner bewusstlos schlagen soll, ohne diesen schwer zu verletzen.
        Der Schlag richtet keine richtigen TP sondern nur TP(A) an.
        Übersteigen die TP(A) die Wundschwelle des Gegners, muss dieser eine KO-Probe ablegen.
        Bei Misslingen, fällt das Opfer für 1W6 SR in Ohnmacht.
        Übersteigen die TP(A) gar die KO, so steht dem Gegner keine KO-Probe zu.
    \item[Besonderes:]
        Kann mit einem \textStyleM{\nameref{aktion.wuchtschlag}} kombiniert werden, die Ansage kann zur TP-Steigerung oder als KO-Erschwernis eingesetzt werden.
\end{description}
