\section{Passierschlag}
\label{aktion.passierschlag}
\textStyleAT{(kein Autotreffer / Freie Raufen oder Ringen-Aktion)}
\begin{description}
    \item[Voraussetzung:]
        Ein Kämpfer durchschreitet den eigenen Kontrollbereich oder gibt sich durch eine Aktion eine Blösse
    \item[Probe:]
        AT+4
    \item[Wirkung:]
        Als Freie Aktion kann einem unvorsichtigen Kämpfer, der den eigenen Kontrollbereich durchschritten oder sich durch ein Manöver in eine unglückliche Lage begeben hat (Sturmangriff) ein ungezielter Schlag versetzt werden (keine Manöver).
        Besitzt das Opfer die SF \textStyleSF{\nameref{sf.aufmerksamkeit}} ist der Schlag um weitere 4 Punkte erschwert, bei der SF \textStyleSF{\nameref{sf.kampfgespuer}} gar um 6 Punkte (4 aus \textStyleSF{\nameref{sf.aufmerksamkeit}} und 2 aus \textStyleSF{\nameref{sf.kampfgespuer}}).
\end{description}
