\subsection{Unterlaufen}
\label{aktion.unterlaufen}
\textStyleAT{(kein Autotreffer / Aktion)}
\begin{description}
    \item[Voraussetzung]:
        Distanzklasse der Waffe des Angreifers ist kürzer als jene des Verteidigers
    \item[Probe]:
        AT+4 (+F)
    \item[Wirkung]:
        Diese Verzweiflungstat erlaubt dem Angreifer eine kurze Verschnaufpause, bis der Verteidiger ihn wieder von sich weggestossen hat.
        Der Angreifer nützt die Kürze seiner Waffe aus und unterläuft den Gegner.
        Die nächste Aktion des Gegners entfällt.
        Dieser Angriff verursacht kein Schaden.
    \item[Besonderes]:
        Kann mit einer \textStyleM{\nameref{aktion.finte}} kombiniert werden.
\end{description}

%------------------------------------------------------------------
